\documentclass[landscape, 10pt]{article}
\usepackage{geometry}
\usepackage{amsmath}
\usepackage{amssymb}
\usepackage{commath}
\usepackage{fixltx2e,graphicx,mathpazo}
\usepackage[T1]{fontenc}
\usepackage[utf8]{inputenc}
\usepackage[dvipsnames]{xcolor}
\usepackage{fancyhdr}
\usepackage{multicol}
\newcommand{\R}{\mathbb{R}}
\newcommand{\N}{\mathbb{N}}
\newcommand{\C}{\mathbb{C}}
\geometry{a4paper, margin=18mm}
\usepackage{titlesec}
\titlespacing*{\section}{0pt}{0mm}{0mm}
\titlespacing*{\subsection}{0pt}{0mm}{0mm}
\definecolor{darkgreen}{RGB}{0,80,0}
\definecolor{myorange}{RGB}{255,128,0}

\setlength{\parindent}{0pt}
\setlength{\parskip}{0pt}
\setlength{\multicolsep}{4.0pt plus 2.0pt minus 1.5pt}

\makeatletter
\g@addto@macro\normalsize{
  \setlength\abovedisplayskip{1pt}
  \setlength\belowdisplayskip{1pt}
  \setlength\abovedisplayshortskip{1pt}
  \setlength\belowdisplayshortskip{1pt}
}
\makeatother

\begin{document}
\pagestyle{fancy}
\fancyhf{}
\rhead{Jonas Degelo}
\lhead{Analysis I\quad FS2020}
\cfoot{\thepage}

\begin{multicols}{3}
% \section{Skript}
\section{$\R$ und $\C$}
              \colorbox{gray}{Ordnungsvollständigkeit:}
                     Seien $A, B\subseteq\R$ s. d.\\
                     \colorbox{SkyBlue}{(i)} 
                            $A\neq\emptyset$ \qquad
                     \colorbox{SkyBlue}{(ii)} 
                            $\forall a\in A \enspace \forall b 
                            \in B \enspace a \leqslant b$ \\
                      Dann: $\exists c\in\R\enspace$ s.d. 
                     $\enspace\forall a\in A\enspace a\leqslant c$
                     $\forall b\in B\enspace c\leqslant b$\\
              \colorbox{BurntOrange}{Korollar 1.1.7} 
              (\textbf{Archimedisches Prinzip})\\
                     Sei \textcolor{NavyBlue}{$x>0 y\in\R$}
                     \quad Dann: \textcolor{NavyBlue}{
                     $\exists n\in \N \quad y\leqslant n*x$}\\
              \colorbox{magenta}{Satz 1.1.8} \textcolor{NavyBlue}{
                     $\forall t\geqslant 0, t\in\R$} hat \textcolor{NavyBlue}{$x^2=t$} 
                     eine Lösung in $\R$\\
              \colorbox{magenta}{Satz 1.1.10} $\forall x,y\in\R$
                     \begin{multicols}{2}
                     \colorbox{SkyBlue}{(i)} $|x|\geqslant 0$ \\
                     \colorbox{SkyBlue}{(ii)} $|xy|=|x||y|$\\
                     \colorbox{SkyBlue}{(iii)} $|x+y|\leqslant |x|+|y|$\\
                     \colorbox{SkyBlue}{(iv)} $|x+y|\geqslant ||x|-|y||$\\
                     \end{multicols}
              \colorbox{magenta}{Satz 1.1.11}
              (\textbf{Young'sche Ungleichung})\\
                     \textcolor{NavyBlue}{
                     $\forall\varepsilon >0, \forall x,y\in\R$} 
                     gilt: \qquad \textcolor{NavyBlue}{
                     $2|xy|\leqslant\varepsilon x^2+\frac{1}{\varepsilon}y^2$}\\
              \colorbox{cyan}{Definition 1.1.12} 
                     Sei $A\subset\R$\\
                     \colorbox{SkyBlue}{(i)}/
                     \colorbox{SkyBlue}{(ii)} 
                            $c\in\R$ ist eine 
                            \textbf{obere/untere Schranke }
                            von $A$ wenn \textcolor{NavyBlue}{
                            $\forall a \in A \enspace a\leqslant / \geqslant c$}. 
                            A ist \textbf{nach oben/unten beschränkt}, 
                            wenn es eine \textbf{obere/untere Schranke} gibt.\\
                     \colorbox{SkyBlue}{(iii)}/\colorbox{SkyBlue}{(iv)} 
                            $m\in\R$ ist ein 
                            \textbf{Maximum/Minimum} von $A$ wenn $m\in A$ und 
                            $m$ \textbf{obere/untere Schranke} von $A$ ist. \\
              \colorbox{magenta}{Satz 1.1.15}  
                     Sei $A\subseteq\R, A\neq\emptyset$ Sei $A$ 
                     nach oben/unten beschränkt. 
                     Dann gibt es eine kleinste obere/ grösste 
                     untere Schranke von A: \quad \textcolor{NavyBlue}{
                     $c:=sup\,A$ / $c:=inf\,A$}  genannt 
                     \textbf{Supremum/Infimum} von $A$ \\
              \colorbox{BurntOrange}{Korollar 1.1.16} 
                     Seien \textcolor{NavyBlue}{
                     $A\subseteq B\subseteq\R$} Wenn B \textbf{nach oben/unten 
                     beschränkt} ist, folgt \textcolor{NavyBlue}{
                     $\sup\,A\leqslant \sup\,B$ / $\inf\,B\leqslant \inf\,A$}\\
              \colorbox{gray}{Konvention:} 
                     Wenn $A$ \textbf{nicht beschränkt ist}, definieren wir 
                     \textcolor{NavyBlue}{$\sup\,A=+\infty$} bzw. 
                     \textcolor{NavyBlue}{$\inf\,A=-\infty$}\\
              \colorbox{magenta}{Satz 1.3.4} 
              (\textbf{Fundamentalsatz der Algebra}) 
                     Sei $n\geqslant 1,\,n\in\N,\,a_j\in\C$ und 
                     \begin{equation*}
                            P(z)=z^n+a_{n-1}z^{n-1}+...+a_0
                     \end{equation*}
                     Dann 
                     \begin{equation*}
                            \exists z_1,...,z_n\in\C,
                     \end{equation*}
                     so dass
                     \begin{equation*}
                            P(z)=(z-z_1)(z-z_2)...(z-z_n)
                     \end{equation*}
\section{Folgen und Reihen}
       \subsection{Grenzwert einer Folge}
              \colorbox{cyan}{Definition 2.1.1} 
                     Eine \textbf{Folge (reeller Zahlen)} ist 
                     eine Abbildung $a:N^{*}\longrightarrow\R$. 
                     Wir schreiben $a_n$ statt $a(n)$ und 
                     bezeichnen eine Folge mit 
                     \textcolor{NavyBlue}{$(a_n)_{n\geqslant 1}$}\\
              \colorbox{yellow}{Lemma 2.1.3} 
                     Sei $(a_n)_{n\geqslant 1}$ eine 
                     Folge. Dann gibt es \textbf{höchstens eine reelle Zahl} 
                     \textcolor{NavyBlue}{$l\in\R$} 
                     \textbf{mit} der Eigenschaft: 
                     \textcolor{NavyBlue}{$\forall\varepsilon >0$} ist die Menge 
                     \textcolor{NavyBlue}{
                     $\{n\in\N:a_n\notin ]l-\varepsilon,\,l+\varepsilon[\}$} 
                     \textbf{endlich}.\\
              \colorbox{cyan}{Definition 2.1.4} 
                     Eine Folge $(a_n)_{n\geqslant 1}$ ist 
                     \textbf{konvergent}, wenn es \textcolor{NavyBlue}{$l\in\R$} 
                     gibt, so dass 
                     \textcolor{NavyBlue}{$\forall\varepsilon >0$} die Menge 
                     \textcolor{NavyBlue}{
                     $\{n\in\N:a_n\notin ]l-\varepsilon,\, l+\varepsilon[\}$} 
                     \textbf{endlich} ist.\\
              \colorbox{yellow}{Lemma 2.1.6} 
                     Sei $(a_n)_{n\geqslant 1}$ eine Folge. 
                     Folgende Aussagen sind äquivalent\\
                     \colorbox{SkyBlue}{(1)} 
                            \textcolor{NavyBlue}{$(a_n)_{n\geqslant 1}$} 
                            \textbf{konvergiert gegen} 
                            \textcolor{NavyBlue}{$l=\lim\limits_{n\to\infty}a_n$}\\
                     \quad\colorbox{SkyBlue}{(2)} 
                            \textcolor{NavyBlue}{
                            $\forall\varepsilon >0\exists N\geqslant 1$}, 
                            so dass \textcolor{
                            NavyBlue}{
                            $|a_n-l|<\varepsilon \quad\forall n\geqslant N$}\\
              \colorbox{magenta}{Satz 2.1.8} 
                     Seien \textcolor{NavyBlue}{
                     $(a_n)_{n\geqslant 1},\,(b_n){n\geqslant 1}$} 
                     \textbf{konvergent} mit 
                     \textcolor{NavyBlue}{$a=\lim\limits_{n\to\infty}a_n,\,
                     b=\lim\limits_{n\to\infty}b_n$}\\
                     \colorbox{SkyBlue}{(1)} 
                            \textcolor{NavyBlue}{$(a_n+b_n)_{n\geqslant 1}$} 
                            ist \textbf{konvergent}: 
                            \textcolor{NavyBlue}{
                            $\lim\limits_{n\to\infty}(a_n+b_n)=a+b$}\\
                     \colorbox{SkyBlue}{(2)} 
                            \textcolor{NavyBlue}{$(a_n\cdot b_n)_{n\geqslant 1}$} 
                            ist \textbf{konvergent}: 
                            \textcolor{NavyBlue}{
                            $\lim\limits_{n\to\infty}(a_n\cdot b_n)=a\cdot b$}\\
                     \colorbox{SkyBlue}{(3)} Sei 
                            \textcolor{NavyBlue}{$\forall n\geqslant 1\enspace b_n\neq 0$} 
                            und \textcolor{NavyBlue}{$b\neq 0$}. 
                            Dann ist 
                            \textcolor{NavyBlue}{$(\frac{a_n}{b_n})_{n\geqslant 1}$} 
                            \textbf{konvergent} und 
                            \textcolor{NavyBlue}{
                            $\lim\limits_{n\to\infty}(\frac{a_n}{b_n})_{n\geqslant 1}
                            =\frac{a}{b}$}\\
                     \colorbox{SkyBlue}{(4)} Wenn 
                            \textcolor{NavyBlue}{$\exists K\geqslant 1$} 
                            mit \textcolor{NavyBlue}{
                            $\forall n\geqslant K: \enspace a_n\leqslant b_n$}, 
                            folgt \textcolor{NavyBlue}{$a\leqslant b$}\\
              \colorbox{Dandelion}{Beispiel 2.1.9} 
                     \textcolor{NavyBlue}{$b\in\mathbb{Z}:\enspace
                     \lim\limits_{n\to\infty}(1+\frac{1}{n})^b=1$}.
                     Das folgt aus
                     \textcolor{NavyBlue}{
                     $\lim\limits_{n\to\infty}(1+\frac{1}{n})=1$}
                     und wiederholter Anwendung von
                     \textbf{Satz 2.1.8 (2) und (3)}.
       \subsection{Satz von Weierstrass}
              \colorbox{cyan}{Definition 2.2.1} 
                     \colorbox{SkyBlue}{(1)[(2)]} 
                     $(a_n)_{n\geqslant 1}$ ist \textbf{monoton
                     wachsend [fallend]} wenn: \textcolor{NavyBlue}{
                     $a_n\leqslant[\geqslant]a_{n+1}\enspace\forall n\geqslant1$}\\
              \colorbox{magenta}{Satz 2.2.2} 
              (\textbf{Weierstrass}) 
                     Sei \textcolor{NavyBlue}{$(a_n)_{n\geqslant 1}$}
                     \textbf{monoton wachsend 
                     [fallend]} und \textbf{nach oben [unten] beschränkt}. Dann 
                     \textbf{konvergiert} 
                     \textcolor{NavyBlue}{$(a_n)_{n\geqslant 1}$} mit 
                     \begin{equation*}
                            \lim\limits_{n\to\infty}a_n=sup\{a_n:n\geqslant 1\}
                     \end{equation*}
                     \begin{equation*}
                            [\lim\limits_{n\to\infty}a_n=inf\{a_n:n\geqslant 1\}]
                     \end{equation*}
              \colorbox{Dandelion}{Beispiel 2.2.3} 
                     Sei \textcolor{NavyBlue}{$a\in\mathbb{Z}$} und 
                     \textcolor{NavyBlue}{$0\leqslant q<1$}.
                     Dann gilt 
                     \begin{equation*}
                            \lim\limits_{n\to\infty}n^aq^n=0.
                     \end{equation*}
                     Wir können annehmen, dass 
                     \textcolor{NavyBlue}{$q>0$}. Sei 
                     \textcolor{NavyBlue}{$x_n=n^aq^n$}, 
                     dann folgt:
                     % \begin{equation*}
                     \begin{align*}
                            x_{n+1}&=(n+1)^aq^{n+1}\\
                                   &=(\frac{n+1}{n})^aq\cdot n^aq^n\\
                                   &=(1+\frac{1}{n})^a\cdot q\cdot x_n.
                     \end{align*}
                     % \end{equation*}
                     Also:
                     \begin{equation*}
                            x_{n+1}=(1+\frac{1}{n})^a\cdot q\cdot x_n.
                     \end{equation*} 
                     Da \textcolor{NavyBlue}{
                     $\lim\limits_{n\to\infty}(1+\frac{1}{n})^a=1$}
                     (\textbf{Beispiel 2.1.9}), gibt es ein 
                     \textcolor{NavyBlue}{$n_0$}, so dass
                     \begin{equation*}
                            (1+\frac{1}{n})^a<\frac{1}{q}\enspace\forall n\geqslant n_0.
                     \end{equation*}
                     Es folgt: 
                     \begin{equation*}
                            x_{n+1}<x_n\enspace\forall n\geqslant n_0.
                     \end{equation*}
                     Da für \textcolor{NavyBlue}{
                     $x_n>0\enspace\forall n\geqslant1$} 
                     die Folge nach \textbf{unten beschränkt} ist und für 
                     \textcolor{NavyBlue}{$n\geqslant n_0$} \textbf{monoton
                     fallend} ist.\\
                     Sei 
                     \begin{align*}{}
                            l&=\lim\limits_{n\to\infty}x_n
                            =\lim\limits_{n\to\infty}x_{n+1}
                            =\lim\limits_{n\to\infty}(1+\frac{1}{n})^a\cdot qx^n\\
                            &=q\cdot\lim\limits_{n\to\infty}x_n
                            =q\cdot l.
                     \end{align*}
                     Also \textcolor{NavyBlue}{$(1-q)\cdot l=0$} woraus 
                     \textcolor{NavyBlue}{$l=0$} folgt.\\
              \colorbox{green}{Bemerkung 2.2.24} 
                     In \textbf{Beispiel 2.2.3} wird zweimal die 
                     folgende einfache Tatsache verwendet: Sei 
                     \textcolor{NavyBlue}{$(a_n)_{n\geqslant1}$} 
                     eine 
                     \textbf{konvergente} Folge mit 
                     \textcolor{NavyBlue}{$\lim\limits_{n\to\infty}a_n=a$}
                     und 
                     \textcolor{NavyBlue}{$k\in\N$}. 
                     Dann ist die durch 
                     \begin{equation*}
                            b_n:=a_{n+k}\enspace n\geqslant1
                     \end{equation*} 
                     definierte Folge \textbf{konvergent} und 
                     \begin{equation*}
                            \lim\limits_{n\to\infty}b_n=a.
                     \end{equation*}
              \colorbox{yellow}{Lemma 2.2.7} 
              (\textbf{Bernoulli Ungleichung}) 
                     \begin{equation*}
                            (1+x)^n\geqslant 1+n\cdot x \quad\forall n\in\N,\, x>-1
                     \end{equation*}
       \subsection{Limes superior und Limes inferior}
              \colorbox{gray}{Limes inferior/ superior:} 
                     Sei \textcolor{NavyBlue}{$(a_n)_{n\geqslant 1}$}
                     eine \textbf{beschränkte Folge}. Sei 
                     \textcolor{NavyBlue}{
                     $\forall n\geqslant1$}:
                     \begin{align*}
                            b_n=inf\{a_k:k\geqslant n\}\\
                            c_n=sup\{a_k:k\geqslant n\}
                     \end{align*}
                     Dann folgt 
                     \textcolor{NavyBlue}{
                     $\forall n\geqslant1\,b_n\leqslant b_{n+1}$} 
                     (\textbf{monoton wachsend}) und 
                     \textcolor{NavyBlue}{
                     $c_n\geqslant c_{n+1}$} 
                     (\textbf{monoton fallend}) 
                     und \textbf{beide Folgen beschränkt}. 
                     Wir definieren: 
                     \begin{align*}
                            \liminf\limits_{n\to\infty}a_n
                            :=\lim\limits_{n\to\infty}b_n\\
                            \limsup\limits_{n\to\infty}a_n
                            :=\lim\limits_{n\to\infty}c_n
                     \end{align*}
       \subsection{Cauchy Kriterium}
              \colorbox{yellow}{Lemma 2.4.1} 
                     $(a_n)_{n\geqslant1}$ \textbf{konvergiert genau dann, wenn} 
                     \textcolor{NavyBlue}{$(a_n)_{n\geqslant1}$} 
                     \textbf{beschränkt} und 
                     \textcolor{NavyBlue}{$\liminf\limits_{n\to\infty}a_n
                     =\limsup\limits_{n\to\infty}a_n$}\\
              \colorbox{magenta}{Satz 2.4.2} 
              (\textbf{Cauchy Kriterium})
                      $(a_n)_{n\geqslant1}$ ist
                     \textbf{genau dann kovergent, wenn} 
                     \textcolor{NavyBlue}{
                     $\forall\varepsilon >0\,\exists N\geqslant1$}, 
                     \begin{equation*}
                            |a_n-a_m|<\varepsilon\quad
                            \forall n,m\geqslant N
                     \end{equation*}
       \subsection{Satz von Bolzano-Wierstrass}
              \colorbox{cyan}{Definition 2.5.1} 
                     Ein \textbf{abgeschlossenes Intervall} 
                     $I\subseteq\R$ ist von der Form
                     \begin{multicols}{2} %\noindent
                     \colorbox{SkyBlue}{(1)} 
                            $[a,b]\, a\leqslant b\in\R$\\
                     \colorbox{SkyBlue}{(2)}
                            $[a,+\infty[\quad a\in\R$\\
                     \colorbox{SkyBlue}{(3)}
                            $]\scalebox{0.75}[1.0]{\(-\)}\infty,a]\quad a\in\R$\\
                     \colorbox{SkyBlue}{(4)} 
                            $]\scalebox{0.75}[1.0]{\(-\)}\infty,+\infty[=\R$
                     \end{multicols}
              \colorbox{green}{Bemerkung 2.5.2} 
                     Ein Intervall $I\subseteq\R$ ist \textbf{genau dann 
                     abgeschlossen, wenn} für jede konvergente Folge 
                     $(a_n)_{n\geqslant1}$ 
                     mit \textcolor{NavyBlue}{
                     $a_n\in I$\quad$\lim\limits_{n\to\infty}a_n\in I$}.\\
              \colorbox{green}{Bemerkung 2.5.3}
                     Seien $I=[a,b],\,J=[c,d]$ mit \\
                     \textcolor{NavyBlue}{
                     $a\leqslant b,\,c\leqslant d,\,a,b,c,d\in\R$}. 
                     Dann ist 
                     \textcolor{NavyBlue}{$I\subseteq J$} \textbf{genau dann, 
                     wenn} \textcolor{NavyBlue}{$c\leqslant a,\,b\leqslant d$}\\
              \colorbox{magenta}{Satz 2.5.5.5} 
              (\textbf{Cauchy-Cantor}) 
                     Sei
                     \begin{equation*}
                            I_1\supseteq I_2\supseteq ...I_n\supseteq 
                            I_{n+1}\supseteq ...
                     \end{equation*}
                     eine Folge 
                     abgeschlossener Intervalle mit 
                     \begin{equation*}
                            \mathcal{L}(I_1)<+\infty
                     \end{equation*}
                     Dann gilt 
                     \begin{equation*}
                            \bigcap_{n\geqslant1}I_n\neq\emptyset.
                     \end{equation*}
                     Falls \textbf{zudem}  
                     \begin{equation*}
                            \lim\limits_{n\to\infty}\mathcal{L}(I_n)=0
                     \end{equation*} 
                     gilt, enthält 
                     \begin{equation*}
                            \bigcap_{n\geqslant1}I_n
                     \end{equation*}
                     \textbf{genau einen Punkt}.\\
              \colorbox{cyan}{Definition 2.5.7} 
                     Eine \textbf{Teilfolge} einer Folge 
                     \textcolor{NavyBlue}{$(a_n)_{n\geqslant1}$}
                     ist eine Folge 
                     \textcolor{NavyBlue}{$(b_n)_{n\geqslant1}$}, 
                     wobei 
                     \begin{equation*}
                            b_n=a_{l(n)}
                     \end{equation*}
                     und 
                     \begin{equation*}
                            l:\N^{*}\longrightarrow\N^*
                     \end{equation*}
                     eine \textbf{Abbildung} mit der Eigenschaft 
                     \begin{equation*}
                            l(n)<l(n+1)\quad\forall n\geqslant1
                     \end{equation*}
              \colorbox{magenta}{Satz 2.5.9} 
              (\textbf{Bolzano-Weierstrass}) 
                     \textbf{Jede beschränkte Folge} 
                     besitzt eine \textbf{konvergente Teilfolge}.
       \subsection{Folgen in $\R^d$ und $\C$}
              \colorbox{cyan}{Definition 2.6.1} 
                     Eine \textbf{Folge in} 
                     \textcolor{NavyBlue}{$\R^d$} 
                     ist eine Abbildung 
                     \textcolor{NavyBlue}{
                     $a:\N^*\longrightarrow\R^d$}. 
                     Wir schreiben $a_n$ statt $a(n)$ 
                     und bezeichnen die Folge mit 
                     \textcolor{NavyBlue}{
                     $(a_n)_{n\geqslant1}$}\\
              \colorbox{cyan}{Definition 2.6.2} 
                     Eine Folge $(a_n)_{n\geqslant1}$ in $\R^d$ 
                     ist \textbf{konvergent}, wenn 
                     \textcolor{NavyBlue}{$\exists a\in\R^d$}, 
                     so dass
                     \textcolor{NavyBlue}{
                     $\forall\varepsilon >0\,\exists N\geqslant1$} 
                     mit 
                     \begin{equation*}
                            \norm{a_n-a}<\varepsilon
                            \qquad\forall n\geqslant N
                     \end{equation*}
              \colorbox{magenta}{Satz 2.6.3} 
                     Sei \textcolor{NavyBlue}{$b=(b_1,...,b_d)$}. 
                     Folgende Aussagen sind äquivalent:
                     \begin{multicols}{2}
                     \colorbox{SkyBlue}{(1)} 
                           \textcolor{NavyBlue}{$\lim\limits_{n\to\infty}a_n=b$}
                     \colorbox{SkyBlue}{(2)} 
                            \textcolor{NavyBlue}{
                            $\lim\limits_{n\to\infty}a_{nj}=b_j$\quad
                            $\forall 1\leqslant j\leqslant d$}
                     \end{multicols}
              \colorbox{green}{Bemerkung 2.6.4} 
                     Sei \textcolor{NavyBlue}{$x=(x_1,...,x_d)$}. \\
                     Dann ist 
                     \textcolor{NavyBlue}{
                     $\forall 1\leqslant j\leqslant d$}
                     \begin{align*}
                            x^2_j\leqslant\sum_{i=1}^{d}x_i^2=\norm{x}^2\leqslant 
                            d\cdot \max\limits_{1\leqslant i\leqslant d}x_i^2
                     \end{align*}
                     woraus 
                     \begin{equation*}
                            |x_j|\leqslant \norm{x}\leqslant
                            \sqrt{d}\cdot\max
                            \limits_{1\leqslant i\leqslant d}|x_i|
                     \end{equation*}
                     folgt.\\
              \colorbox{green}{Bemerkung 2.6.5} \textbf{Eine konvergente Folge} $(a_n)_{n\geqslant1}$ in 
                     $\R^d$ \textbf{ist beschränkt}. Das heisst: 
                     \textcolor{NavyBlue}{$\exists R\geqslant0$} mit 
                     \textcolor{NavyBlue}{
                     $\norm{a_n}\leqslant R\enspace\forall n\geqslant1$}\\
              \\
              \colorbox{magenta}{Satz 2.6.6} 
                     \colorbox{SkyBlue}{(1)} 
                            Eine Folge $(a_n)_{n\geqslant1}$ 
                            \textbf{konvergiert genau dann, wenn} sie eine 
                            \textbf{Cauchy Folge} ist: 
                            \textcolor{NavyBlue}{
                            $\forall\varepsilon >0\,\exists N\geqslant1$} mit 
                            \textcolor{NavyBlue}{
                            $\norm{a_n-a_m}<\varepsilon$
                            \quad$\forall n,m\geqslant N$}.\\
                     \colorbox{SkyBlue}{(2)} 
                            \textbf{Jede beschränkte Folge} hat eine 
                            \textbf{konvergente Teilfolge}.\\
       \subsection{Reihen}
              \colorbox{cyan}{Definition 2.7.1} 
                     Die \textbf{Reihe 
                     $\sum_{k=1}^\infty a_k$ ist konvergent}, 
                     wenn die Folge 
                     \textcolor{NavyBlue}{$(S_n)_{n\geqslant1}$} 
                     der \textbf{Partialsummen konvergiert}. In 
                     diesem Fall definieren wir: 
                     \textcolor{NavyBlue}{
                     $\sum_{k=1}^\infty a_k
                     =\lim\limits_{n\to\infty}S_n$}\\
              \colorbox{Dandelion}{Beispiel 2.7.2} 
              (\textbf{Geometrische Reihe}) 
                     Sei $q\in\C$ mit 
                     $|q|<1$. Dann konvergiert
                     \begin{equation*}
                            \sum_{k=0}^\infty q^k=\frac{1}{1-q}
                     \end{equation*}
                     Sei 
                     \begin{align*}
                            S_n=\sum_{k=0}^n q^k=1+q+...+q^n.\\
                            q\cdot S_n=q+...+q^n+q^{n+1}
                     \end{align*}
                     woraus
                     \begin{equation*}
                            (1-q)S_n=1-q^{n+1}
                     \end{equation*}
                     folgt. Es gilt also: 
                     \begin{equation*}
                            S_n=\frac{1-q^{n+1}}{1-q}
                     \end{equation*}
                     Nun zeigen wir die Konvergenz: 
                     \begin{equation*}
                            |S_n-\frac{1}{1-q}|
                            =|\frac{-q^{n+1}}{1-q}
                            =\frac{|q|^{n+1}}{|1-q|}.
                     \end{equation*}
                     Es folgt aus
                     \textbf{Beispiel 2.2.3} 
                     und $0\leqslant|q|<1$:
                     \begin{equation*}
                            \lim\limits_{n\to\infty}|S_n-\frac{1}{1-1}|
                            =\lim\limits_{n\to\infty}\frac{|q|^{n+1}}{|1-q|}=0.
                     \end{equation*}
                     Somit konvergiert $(S_n)_{n\geqslant1}$ gegen
                     $\frac{1}{1-q}$.\\
              \colorbox{Dandelion}{Beispiel 2.7.3} 
              (\textbf{Harmonische Reihe}) 
                     Die Reihe
                     \begin{equation*}
                            \sum_{n=1}^\infty\frac{1}{n}
                     \end{equation*}
                     divergiert.
              \newpage
              \colorbox{magenta}{Satz 2.7.4} Seien 
                     \textcolor{NavyBlue}{
                     $\sum_{k=1}^\infty a_k,\,\sum_{j=1}^\infty b_j$} 
                     \textbf{konvergent} sowie 
                     \textcolor{NavyBlue}{$\alpha\in\C$}. 
                     Dann ist:\\
                     \colorbox{SkyBlue}{(1)} 
                            \textcolor{NavyBlue}{
                            $\sum_{k=1}^\infty(a_k+b_k)$} 
                            \textbf{konvergent} und 
                            \begin{equation*}
                                   \sum_{k=1}^\infty(a_k+b_k)
                                   =(\sum_{k=1}^\infty a_k)
                                   +(\sum_{j=1}^\infty b_j)
                            \end{equation*}
                     \colorbox{SkyBlue}{(2)} 
                            \textcolor{NavyBlue}{
                            $\sum_{k=1}^\infty(\alpha\cdot a_k)$} 
                            \textbf{konvergent} und 
                            \begin{equation*}
                                   \sum_{k=1}^\infty
                                   (\alpha\cdot a_k)
                                   =\alpha\cdot
                                   \sum_{k=1}^\infty a_k
                            \end{equation*}
              \colorbox{magenta}{Satz 2.7.5} 
              (\textbf{Cauchy Kriterium}) 
                     Die Reihe 
                     \textcolor{NavyBlue}{
                     $\sum_{k=1}^\infty a_k$ }
                     ist \textbf{genau dann konvergent, wenn}:
                     \textcolor{NavyBlue}{
                     $\forall\varepsilon >0\,
                     \exists N\geqslant1$} mit 
                     \textcolor{NavyBlue}{
                     $|\sum_{k=n}^m a_k|<\varepsilon$\quad
                     $\forall m\geqslant n\geqslant N$}.\\
              \colorbox{magenta}{Satz 2.7.6} 
                     Eine Reihe \textcolor{NavyBlue}{
                     $\sum_{k=1}^\infty a_k$} mit 
                     \begin{equation*}
                            a_k\geqslant0\quad\forall k\in\N^*
                     \end{equation*}
                     \textbf{konvergiert genau dann, wenn} 
                     die Folge 
                     \textcolor{NavyBlue}{
                     $(S_n)_{n\geqslant1}$}
                     \begin{equation*}
                            S_n=\sum_{k=1}^n a_k
                     \end{equation*}
                     der \textbf{Partialsummen nach 
                     oben beschränkt ist}.\\
              \colorbox{BurntOrange}{Korollar 2.7.7} 
              (\textbf{Vergleichssatz}) 
                     Seien \textcolor{NavyBlue}{
                     $\sum_{k=1}^\infty a_k$}, 
                     \textcolor{NavyBlue}{
                     $\sum_{k=1}^\infty b_k$} 
                     Reihen mit: 
                     \textcolor{NavyBlue}{
                     $0\leqslant a_k\leqslant b_k\quad
                     \forall k\geqslant1$}. 
                     Dann gelten:
                     \begin{align*}
                            \sum_{k=1}^\infty b_k\,
                            \textbf{konvergent}
                            &\Longrightarrow 
                            \sum_{k=1}^\infty a_k\,
                            \textbf{konvergent}\\
                            \sum_{k=1}^\infty a_k\,
                            \textbf{divergent} 
                            &\Longrightarrow 
                            \sum_{k=1}^\infty b_k\,
                            \textbf{divergent} 
                     \end{align*}
                     Die \textbf{Implikationen treffen auch zu}, 
                     wenn \textcolor{NavyBlue}{
                     $\exists K\geqslant1$} mit 
                     \begin{equation*}
                            0\leqslant a_k\leqslant b_k\quad
                            \forall k\geqslant K
                     \end{equation*}
              \colorbox{cyan}{Definition 2.7.9} 
                     Die Reihe 
                     \textcolor{NavyBlue}{
                     $\sum_{k=1}^\infty a_k$} ist 
                     \textbf{absolut konvergent}, 
                     wenn \textcolor{NavyBlue}{
                     $\sum_{k=1}^\infty |a_k|$} 
                     \textbf{konvergiert}.\\
              \colorbox{magenta}{Satz 2.7.10} 
                     Eine \textbf{absolut konvergente Reihe} 
                     \textcolor{NavyBlue}{$\sum_{k=1}^\infty a_k$}
                     ist \textbf{auch konvergent} und es gilt:
                     \textcolor{NavyBlue}{
                     $|\sum_{k=1}^\infty a_k|
                     \leqslant\sum_{k=1}^\infty |a_k|$}\\
              \colorbox{magenta}{Satz 2.7.12} 
              (\textbf{Leibniz} 1682) 
                     Sei \textcolor{NavyBlue}{
                     $(a_n)_{n\geqslant1}$}
                     \textbf{monoton fallend} 
                     mit \textcolor{NavyBlue}{
                     $a_n\geqslant0\quad\forall n\geqslant1$} und 
                     \textcolor{NavyBlue}{
                     $\lim\limits_{n\to\infty}a_n=0$}. 
                     Dann \textbf{konvergiert} 
                     \begin{equation*}
                            S:=\sum_{k=1}^\infty (-1)^{k+1}a_k
                     \end{equation*}
                     und es gilt: 
                     \begin{equation*}
                            a_1-a_2\leqslant S\leqslant a_1
                     \end{equation*}
              \colorbox{cyan}{Definition 2.7.14} 
                     Eine Reihe \textcolor{NavyBlue}{$\sum_{n=1}^\infty a_n'$} 
                     ist eine 
                     \textbf{Umordnung der Reihe} 
                     \textcolor{NavyBlue}{$\sum_{n=1}^\infty a_n$} 
                     wenn es eine \textbf{bijektive Abbildung} 
                     \textcolor{NavyBlue}{$\phi :\N^*\longrightarrow\N^*$} 
                     \textbf{gibt}, so dass 
                     \textcolor{NavyBlue}{$a_n'=a_{\phi(n)}$}\\
              \colorbox{magenta}{Satz 2.7.16} 
              (Drichlet 1837) Wenn 
                     \textcolor{NavyBlue}{$\sum_{n=1}^\infty a_n$} \textbf{absolut 
                     konvergiert, dann konvergiert jede Umordnung} der Reihe 
                     mit \textbf{demselben Grenzwert}.\\
              \colorbox{magenta}{Satz 2.7.17} 
              (\textbf{Quotientenkriterium}, Cauchy 1821) 
                     Sei \textcolor{NavyBlue}{
                     $(a_n)_{n\geqslant1}$} mit 
                     \textcolor{NavyBlue}{
                     $a_n\neq0\quad\forall n\geqslant1$}
                     und \textcolor{NavyBlue}{
                     $\sum_{n=1}^\infty a_n$} eine Reihe.
                     \begin{align*}
                            \limsup\limits_{n\to\infty}
                            \frac{|a_{n+1}|}{|a_n|}<1 \,
                            &\Longrightarrow
                            \sum_{n=1}^\infty a_n \,
                            \textbf{konvergiert absolut} \\
                            \liminf\limits_{n\to\infty}
                            \frac{|a_{n+1}|}{|a_n|}>1 \,
                            &\Longrightarrow
                            \sum_{n=1}^\infty a_n \,
                            \textbf{divergiert}
                     \end{align*}
              \colorbox{Dandelion}{Beispiel 2.7.18} 
              (\textbf{Exponentialfunktion})\\
                     Für \textcolor{NavyBlue}{$z\in\C$}
                     betrachten wir die Reihe:
                     \begin{equation*}
                            1+z+\frac{z^2}{2!}+\frac{z^3}{3!}+\dots
                     \end{equation*}
                     mit allgemeinem Glied
                     \begin{equation*}
                            a_n=\frac{z^n}{n!}.
                     \end{equation*}
                     Dann folgt für \textcolor{NavyBlue}{$z\neq0$}:
                     \begin{equation*}
                            \frac{|a_{n+1}|}{|a_n|}
                            =|\frac{z^{n+1}}{(n+1)!}\frac{n!}{z^n}|
                            =\frac{|z|}{n+1}.
                     \end{equation*}
                     Also gilt:
                     \begin{equation*}
                            \lim\limits_{n\to\infty}\frac{|a_{n+1}|}{|a_n|}=0
                     \end{equation*}
                     und die Reihe konvergiert 
                     für alle \textcolor{NavyBlue}{$z\in\C$}.
                     Wir definieren die \textbf{Exponentialfunktion}: 
                     \begin{equation*}
                            \exp z:=1+z+\frac{z^2}{2!}+\frac{z^3}{3!}+\dots
                            =\sum_{n=0}^\infty\frac{z^n}{n!}
                     \end{equation*}
              \colorbox{green}{Bemerkung 2.7.19} 
                     Das \textbf{Quotientenkriterium versagt}, wenn z. B. 
                     \textbf{unenedlich viele Glieder} $a_n$ der Reihe 
                     \textbf{verschwinden} ($=0$ sind)\\
              \vfill\null
              \columnbreak
              \colorbox{magenta}{Satz 2.7.20}
              (\textbf{Wurzelkriterium}, Cauchy 1821)\\
                     \begin{align*}
                            \limsup\limits_{n\to\infty}
                            \sqrt[n]{|a_n|}<1
                            &\Longrightarrow
                            \sum_{n=1}^\infty a_n \,
                            \textbf{konvergiert absolut}\\
                            \limsup\limits_{n\to\infty}
                            \sqrt[n]{|a_n|}>1
                            &\Longrightarrow
                            \sum_{n=1}^\infty a_n \, 
                            \text{und} \,
                            \sum_{n=1}^\infty |a_n| \,
                            \textbf{divergieren}
                     \end{align*}
              \colorbox{gray}{Konvergenzradius} \\
                     Sei \textcolor{NavyBlue}{
                     $(c_k)_{k\geqslant0}$} 
                     eine Folge (in $\R$ oder $\C$). Wenn 
                     \begin{equation*}
                            \limsup\limits_{k\to\infty}
                            \sqrt[k]{|c_k|}
                     \end{equation*}
                     existiert, definieren wir: 
                     \begin{align*}
                            \rho&=+\infty
                            &\text{ wenn } 
                            \limsup\limits_{k\to\infty}
                            \sqrt[k]{|c_k|}=0\\
                            \rho&=\frac{1}{\limsup
                            \limits_{k\to\infty}\sqrt[k]{|c_k|}}
                            &\text{ wenn }
                            \limsup\limits_{k\to\infty}
                            \sqrt[k]{|c_k|} >0
                     \end{align*}
              \colorbox{gray}{Riemann Zeta Funktion} Sei $s>1$ und 
                     \begin{equation*}
                            \zeta(s)=\sum_{n=1}^\infty\frac{1}{n^s}.
                     \end{equation*}
                     Wir wissen, dass
                     \textcolor{NavyBlue}{
                     $\sum_{n=1}^\infty\frac{1}{n^2}$} konvergiert.
                     Die Reihe konvergiert 
                     \textcolor{NavyBlue}{$\forall s>1$}\\
              \colorbox{BurntOrange}{Korollar 2.7.21}
                     Die \textbf{Potenzreihe}
                     \textcolor{NavyBlue}{$\sum_{k=0}^\infty c_kz^k$}
                     \textbf{konvergiert absolut} 
                     \textcolor{NavyBlue}{$\forall |z|<\rho$} und \textbf{divergiert} 
                     \textcolor{NavyBlue}{$\forall |z|>\rho$}.\\
              \colorbox{cyan}{Definition 2.7.22} 
                     \textcolor{NavyBlue}{$\sum_{k=0}^\infty b_k$}
                     ist eine \textbf{lineare Anordnung der Doppelreihe} 
                     \textcolor{NavyBlue}{$\sum_{i,j\geqslant0}a_{ij}$}, 
                     wenn es eine 
                     \textbf{Bijektion} 
                     \textcolor{NavyBlue}{$\sigma:\N\longrightarrow\N\times\N$}
                     \textbf{gibt}, mit 
                     \textcolor{NavyBlue}{$b_k=a_{\sigma(k)}$}.\\
              \colorbox{magenta}{Satz 2.7.23} 
              (Cauchy 1821) 
                     Wir nehmen an, dass es 
                     \textcolor{NavyBlue}{$B\geqslant0$} gibt, so dass 
                     \textcolor{NavyBlue}{
                     $\sum_{i=0}^m\sum_{j=0}^m|a_{ij}|\leqslant B$
                     \quad
                     $\forall m\geqslant0$}.
                     Dann 
                     \textbf{kovergieren die folgenden Reihen absolut}: 
                     \begin{align*}
                            S_i:=\sum_{j=0}^\infty a_{ij} \quad
                            \forall i\geqslant0
                            \text{ und }
                            U_j:=\sum_{i=0}^\infty a_{ij} \quad
                            \forall j\geqslant0
                     \end{align*}
                     sowie 
                     \begin{align*}
                            \sum_{i=0}^\infty S_i
                            \text{ und }
                            \sum_{j=0}^\infty U_j
                     \end{align*}
                     und es gilt:
                     \begin{align*}
                            \sum_{i=0}^\infty S_i
                            =\sum_{j=0}^\infty U_j
                     \end{align*}
                     Und \textbf{jede lineare Anordnug 
                     der Doppelreihe konvergiert 
                     absolut mit gleichem Grenzwert}.\\
              \colorbox{cyan}{Definition 2.7.24} 
                     Das \textbf{Cauchy Produkt} der Reihen 
                     \textcolor{NavyBlue}{
                     $\sum_{i=0}^\infty a_i,\,\sum_{j=0}^\infty b_j$} 
                     ist die Reihe: 
                     \begin{align*}
                            &\sum_{n=0}^\infty
                            (\sum_{j=0}^\infty a_{n-j}b_j)\\
                            &=a_0b_0+(a_0b_1+a_1b_0)
                            +(a_0b_2+a_1b_1+a_2b_0)+...
                     \end{align*}
              \colorbox{magenta}{Satz 2.7.26} Falls die Reihen 
                     \textcolor{NavyBlue}{
                     $\sum_{i=0}^\infty a_i,\,\sum_{j=0}^\infty b_j$}
                     \textbf{absolut konvergieren}, 
                     \textbf{konvergiert ihr Cauchy Produkt}
                     und es gilt: 
                     \begin{align*}
                            \sum_{n=0}^\infty
                            (\sum_{j=0}^\infty a_{n-j}b_j)
                            =(\sum_{i=0}^\infty a_i)
                            (\sum_{j=0}^\infty b_j)
                     \end{align*}
              \colorbox{Dandelion}{Anwendung 2.7.27} 
              (\textbf{Exponentialfunktion})
                     \textcolor{NavyBlue}{
                     $\forall z,w\in\C$}
                     \begin{equation*}
                            \exp(w+z)=\exp(w)\exp(z).                            
                     \end{equation*}
                     Wir berechnen das Cauchy Produkt der 
                     Reihen: \textcolor{NavyBlue}{
                     $\sum_{i=0}^\infty\frac{w^i}{i!}$},
                     \textcolor{NavyBlue}{
                     $\sum_{i=0}^\infty\frac{z^i}{i!}$}.
                     Dieses ist:
                     \textcolor{NavyBlue}{
                     $\sum_{n=0}^\infty(\sum_{j=0}^n
                     \frac{w^{n-j}}{(n-j)!}
                     \frac{z^j}{j!})$} Woraus die Behauptung folgt.\\
              \colorbox{magenta}{Satz 2.7.28} 
                     Sei \textcolor{NavyBlue}{$f_n:\N\longrightarrow\R$} 
                     eine Folge. Wir nehmen an, dass: \\
                     \colorbox{SkyBlue}{(1)} 
                            \textcolor{NavyBlue}{
                            $f(j):=\lim\limits_{n\to\infty}f_n(j)$\quad
                            $\forall j\in\N$} 
                            existiert \\
                     \colorbox{SkyBlue}{(2)} es eine Funktion 
                            \textcolor{NavyBlue}{
                            $g:\N\longrightarrow[0,\infty[$} gibt,
                            so dass \\
                            \colorbox{SeaGreen}{2.1} 
                                   \textcolor{NavyBlue}{
                                   $|f_n(j)\leqslant g(j)$\quad
                                   $\forall j\geqslant0,\enspace
                                   \forall n\geqslant0$} \\
                            \colorbox{SeaGreen}{2.2}
                                   \textcolor{NavyBlue}{$\sum_{j=0}^\infty g(j)$} 
                                   \textbf{konvergiert}. \\
                            Dann folgt: 
                            \begin{equation*}
                                   \sum_{j=0}^\infty f(j)
                                   =\lim\limits_{n\to\infty}
                                   \sum_{j=0}^\infty f_n(j)
                            \end{equation*}
              \colorbox{BurntOrange}{Korollar 2.7.29} 
              (\textbf{Exponentialfunktion})
                     \textcolor{NavyBlue}{
                     $\forall z\in\C$} \textbf{konvergiert} die Folge 
                     \textcolor{NavyBlue}{
                     $((1+\frac{z}{n})^n)_{n\geqslant1}$} und 
                     \begin{equation*}
                            \lim\limits_{n\to\infty}
                            (1+\frac{z}{n})^n=exp(z)
                     \end{equation*}

\section{Stetige Funktionen}
       \subsection{Reellwertige Funktionen}
              \colorbox{cyan}{Definition 3.1.1} 
                     $f$ ist nach 
                     \textbf{[oben/unten] beschränkt} wenn 
                     \textcolor{NavyBlue}{$f(D)\subseteq\R$} 
                     nach \textbf{[oben/unten] beschränkt} ist.\\
              \colorbox{cyan}{Definition 3.1.2} 
                     Eine Funktion $f:D\longrightarrow\R$, wobei 
                     $D\subseteq\R$, ist:\\
                     \colorbox{SkyBlue}{(1)[(2)]}
                            \textbf{[streng] monoton 
                            wachsend}, wenn 
                            \begin{equation*}
                                   \forall x,y\in D\quad 
                                   x\leqslant [<]y\Rightarrow
                                   f(x)\leqslant [<]f(y)
                            \end{equation*}
                     \colorbox{SkyBlue}{(3)[(4)]}
                            \textbf{[streng] monoton 
                            fallend}, wenn 
                            \begin{equation*}
                                   \forall x,y\in D\quad 
                                   x\leqslant [<]y\Rightarrow
                                   f(x)\geqslant [>]f(y)
                            \end{equation*}
                     \colorbox{SkyBlue}{(5)[(6)]} 
                            \textbf{[streng] monoton}, 
                            wenn \textcolor{NavyBlue}{$f$} 
                            \textbf{[streng] monoton wachsend} oder 
                            \textbf{fallend} ist.\\
       \subsection{Stetigkeit}
              \colorbox{cyan}{Definition 3.2.1} 
                     Sei $D\subseteq\R,\,x_0\in D$. 
                     Die Funktion \textcolor{NavyBlue}{$f:D\longrightarrow\R$} 
                     ist in $x_0$ \textbf{stetig}, 
                     wenn 
                     \begin{equation*}
                            \forall\varepsilon>0
                            \enspace\exists\delta>0
                     \end{equation*}
                     so dass 
                     \textcolor{NavyBlue}{$\forall x\in D$} 
                     die Implikation:
                     \begin{equation*}
                            |x-x_0|<\delta\Rightarrow|f(x)-f(x_0)|<\varepsilon
                     \end{equation*}
                     gilt\\
              \colorbox{cyan}{Definition 3.2.2} 
                     Die Funktion \textcolor{NavyBlue}{$f:D\longrightarrow\R$}
                     ist \textbf{stetig}, wenn sie in 
                     \textbf{jedem Punkt von D stetig} ist.\\
              \colorbox{magenta}{Satz 3.2.4} Sei 
                     \textcolor{NavyBlue}{$x_0\in D\subseteq\R$} und 
                     \textcolor{NavyBlue}{$f:D\longrightarrow\R$}. 
                     Die Funktion \textcolor{NavyBlue}{$f$} ist genau 
                     dann in \textcolor{NavyBlue}{$x_0$} 
                     \textbf{stetig}, wenn für 
                     \textbf{jede Folge} 
                     \textcolor{NavyBlue}{$(a_n)_{n\geqslant1}$} 
                     in $D$ die folgende Implikation gilt: 
                     \begin{equation*}
                            \lim\limits_{n\to\infty}a_n=x_0\Longrightarrow
                            \lim\limits_{n\to\infty}f(a_n)=f(x_0). 
                     \end{equation*}
              \colorbox{BurntOrange}{Korollar 3.2.5} Seien 
                     \textcolor{NavyBlue}{
                     $x_0\in D\subseteq\R,\,\lambda\in\R$} und \\
                     \textcolor{NavyBlue}{
                     $f:D\longrightarrow\R$},\,
                     \textcolor{NavyBlue}{
                     $g:D\longrightarrow\R$}
                     beide \textbf{stetig} in \textcolor{NavyBlue}{$x_0$}:\\
                     \colorbox{SkyBlue}{(1)}
                            \textcolor{NavyBlue}{
                            $f+g,\,\lambda\cdot f,\,f\cdot g$} 
                            sind \textbf{stetig} in 
                            \textcolor{NavyBlue}{$x_0$}.\\
                     \colorbox{SkyBlue}{(2)}
                            Wenn \textcolor{NavyBlue}{$g(x_0)\neq0$}
                            ist, ist 
                            \begin{align*}
                                   \frac{f}{g}:D\cap\{x\in D:g(x)\neq0\}
                                   \longrightarrow&\R\\
                                   x\longmapsto&\frac{f(x)}{g(x)}
                            \end{align*}
                            \textbf{stetig} in \textcolor{NavyBlue}{$x_0$}\\

              \colorbox{cyan}{Definition 3.2.6} 
                     Eine \textbf{polynomielle Funktion} \\
                     \textcolor{NavyBlue}{$P:\R\longrightarrow\R$} 
                     ist eine Funktion 
                     der Form: 
                     \begin{equation*}
                            P(x)=a_nx^n+...+a_0
                     \end{equation*}
                     wobei: \textcolor{NavyBlue}{$a_n,...,a_0\in\R$}. 
                     Wenn \textcolor{NavyBlue}{$a_n\neq0$} ist, ist 
                     \textcolor{NavyBlue}{$n$} 
                     der \textbf{Grad} von \textcolor{NavyBlue}{$P$}.\\
              \colorbox{BurntOrange}{Korollar 3.2.7} 
                     \textbf{Polynomielle Funktionen} 
                     sind auf ganz $\R$ \textbf{stetig}.\\
              \colorbox{BurntOrange}{Korollar 3.2.8} 
                     Seien $P,Q$ \textbf{polynomielle Funktionen} 
                     auf $\R$ mit \textcolor{NavyBlue}{$Q\neq0$}. Seien 
                     \textcolor{NavyBlue}{$x_1,...,x_m$} die Nullstellen von 
                     \textcolor{NavyBlue}{$Q$}. 
                     Dann ist 
                     \begin{align*}
                            \frac{P}{Q}:\R\setminus
                            \{x_1,...,x_m\}\longrightarrow&\R \\
                            x\longmapsto&\frac{P(x)}{Q(x)}
                     \end{align*}
                     \textbf{stetig}.\\
       \subsection{Zwischenwertsatz}
              \colorbox{magenta}{Satz 3.3.1} (\textbf{Zwischenwertsatz}, Bolzano 1817) Seien 
                     \textcolor{NavyBlue}{$I\subseteq\R$} ein Intervall, 
                     \textcolor{NavyBlue}{$f:I\longrightarrow\R$} 
                     eine \textbf{stetige Funktion} und 
                     \textcolor{NavyBlue}{$a,b\in I$}. 
                     \textbf{Für jedes} \textcolor{NavyBlue}{$c$} \textbf{zwischen} 
                     \textcolor{NavyBlue}{$f(a)$} und \textcolor{NavyBlue}{$f(b)$} 
                     \textbf{gibt es ein} \textcolor{NavyBlue}{$z$} 
                     zwischen \textcolor{NavyBlue}{$a$} und \textcolor{NavyBlue}{$b$} mit 
                     \textcolor{NavyBlue}{$f(z)=c$}.
       \subsection{Min-Max Satz}
              \colorbox{cyan}{Definition 3.4.2}
                     Ein \textbf{Intervall} $I\subset\R$ ist 
                     \textbf{kompakt}, wenn es 
                     von der Form 
                     \textcolor{NavyBlue}{$I=[a,b],\quad a\leqslant b$} ist.\\
              \colorbox{yellow}{Lemma 3.4.3} 
                     Sei \textcolor{NavyBlue}{$D\subseteq\R,\,x_0\in D$} und 
                     \textcolor{NavyBlue}{$f,g:D\longrightarrow\R$} 
                     \textbf{stetig} in \textcolor{NavyBlue}{$x_0$}. So sind 
                     \textcolor{NavyBlue}{
                     $|f|,\,\max(f,g),\,\min(f,g)$} \textbf{stetig} in 
                     \textcolor{NavyBlue}{$x_0$}.\\
              \colorbox{yellow}{Lemma 3.4.4} 
                     Sei \textcolor{NavyBlue}{$(x_n)_{n\geqslant1}$}
                     eine \textbf{konvergente Folge} in 
                     $\R$ mit Grenzwert 
                     \textcolor{NavyBlue}{$\lim\limits_{n\to\infty}x_n\in\R$}. 
                     Sei \textcolor{NavyBlue}{$a\leqslant b$}. 
                     Wenn \textcolor{NavyBlue}{$\{x_n:n\geqslant1\}\subseteq[a,b]$}, folgt: 
                     \textcolor{NavyBlue}{
                     $\lim\limits_{n\to\infty}x_n\in[a,b]$}. \\
              \colorbox{magenta}{Satz 3.4.5} Sei \textcolor{NavyBlue}{$f:I=[a,b]\longrightarrow\R$}
                     \textbf{stetig auf einem kompakten Intervall} \textcolor{NavyBlue}{$I$}. 
                     Dann gibt es \textcolor{NavyBlue}{$u\in I$} und 
                     \textcolor{NavyBlue}{$v\in I$} mit: 
                     \textcolor{NavyBlue}{
                     $f(u)\leqslant f(x)\leqslant f(v)\quad\forall x\in I$}. 
                     \textbf{Insbesondere ist 
                     \textcolor{NavyBlue}{$f$} beschränkt}.
       \subsection{Umkehrabbildung}
              \colorbox{magenta}{Satz 3.5.1}
                     Seien \textcolor{NavyBlue}{$D_1,D_2\subseteq\R$, 
                     $f:D_1\longrightarrow D_2,\,g:D_2\longrightarrow\R$} und 
                     \textcolor{NavyBlue}{$x_0\in D_1$}.
                     Wenn \textcolor{NavyBlue}{$f$} in 
                     \textcolor{NavyBlue}{$x_0$} und 
                     \textcolor{NavyBlue}{$g$} in
                     \textcolor{NavyBlue}{$f(x_0)$} 
                     \textbf{stetig sind}, 
                     so ist \textcolor{NavyBlue}{
                     $g\circ f:D_1\longrightarrow\R$} in
                     \textcolor{NavyBlue}{$x_0$} 
                     \textbf{stetig}.\\
              \colorbox{BurntOrange}{Korollar 3.5.2} 
                     Wenn in \textbf{Satz 3.5.1} 
                     \textcolor{NavyBlue}{$f$} auf 
                     \textcolor{NavyBlue}{$D_1$} und 
                     \textcolor{NavyBlue}{$g$} auf \textcolor{NavyBlue}{$D_2$}
                     stetig sind, ist \textcolor{NavyBlue}{$g\circ f$} auf 
                     \textcolor{NavyBlue}{$D_1$}
                     \textbf{stetig}.\\
              \colorbox{magenta}{Satz 3.5.3} 
                     Sei \textcolor{NavyBlue}{$I\subseteq\R$} 
                     ein Intervall und 
                     \textcolor{NavyBlue}{$f:I\longrightarrow\R$} 
                     \textbf{stetig, streng monoton}. Dann ist 
                     \textcolor{NavyBlue}{$J:=f(i)\subseteq\R$} ein Intervall 
                     und \textcolor{NavyBlue}{$f^{-1}:J\longrightarrow I$} 
                     ist \textbf{stetig, streng monoton}.
       \subsection{Reelle Exponentialfunktion}
              \colorbox{magenta}{Satz 3.6.1} 
                     \textcolor{NavyBlue}{
                     $\exp:\R\longrightarrow]0,+\infty[$}
                     ist \textbf{streng monoton wachsend, stetig und surjektiv}.\\
              \colorbox{BurntOrange}{Korollar 3.6.2} 
                     \begin{equation*}
                            \exp(x)>0\quad\forall x\in\R.
                     \end{equation*}
                     Aus der \textbf{Potenzreihendarstellung} von 
                     \textcolor{NavyBlue}{$\exp$} folgt ausserdem: 
                     \begin{equation*}
                            \exp(x)>1\quad\forall x>0.
                     \end{equation*} 
                     Wenn \textcolor{NavyBlue}{$y<z$} ist, folgt 
                     (aus \textbf{2.7.27})
                     \begin{align*}
                            \exp(z)&=\exp(y+(z-y))\\
                                   &=\exp(y)\exp(z-y)
                     \end{align*}
                     und da \textcolor{NavyBlue}{$\exp(z-y)>1$} 
                     ist folgt folgendes Korollar:\\
              \colorbox{BurntOrange}{Korollar 3.6.3} 
                     \textcolor{NavyBlue}{
                     $\exp(z)>exp(y)\quad\forall z>y$}\\
              \colorbox{BurntOrange}{Korollar 3.6.4} 
                     \textcolor{NavyBlue}{
                     $\exp(x)\geqslant1+x\quad\forall x\in\R$}\\
              \colorbox{BurntOrange}{Korollar 3.6.5} 
                     Der \textbf{natürlich Logarithmus} 
                     $\ln:]0,+\infty[\longrightarrow\R$ ist eine 
                     \textbf{streng monoton 
                     wachsende, stetige, bijektive} 
                     Funktion. Des weiteren gilt 
                     \begin{equation*}
                            \ln(a\cdot b)=\ln a+\ln b\quad
                            \forall a,b\in]0,+\infty[.
                     \end{equation*}
              \colorbox{BurntOrange}{Korollar 3.6.6} 
                     \colorbox{SkyBlue}{(1)/(2)}
                            Für \textcolor{NavyBlue}{$a>$/$<0$} 
                            ist 
                            \begin{align*}
                                   ]0,+\infty[\longrightarrow
                                   &]0,+\infty[\\
                                   x\longmapsto &x^a
                            \end{align*}
                            eine \textbf{stetige, streng monoton 
                            wachsende/fallende Bijektion}. \\
                            \textcolor{NavyBlue}{
                            $\forall a,b\in\R,\,\forall x>0$}:\\
                     \colorbox{SkyBlue}{(3)} 
                            \textcolor{NavyBlue}{
                            $\ln(x^a)=a\ln(x)$} \qquad
                     \colorbox{SkyBlue}{(4)} 
                            \textcolor{NavyBlue}{
                            $x^a\cdot x^b=x^{a+b}$}\\
                     \colorbox{SkyBlue}{(5)} 
                            \textcolor{NavyBlue}{
                            $(x^a)^b=x^{a\cdot b}$}
       \subsection{Konvergenz von Funktionenfolgen}
              \colorbox{cyan}{Definition 3.7.1} 
                     Die \textbf{Funktionenfolge} 
                     \textcolor{NavyBlue}{$(f_n)_{n\geqslant0}$}
                     \textbf{konvergiert punktweise} 
                     gegen eine Funktion
                     \textcolor{NavyBlue}{
                     $f:D\longrightarrow\R$}, wenn 
                     \begin{equation*}
                            \forall x\in D:\,
                            f(x)
                            =\lim\limits_{n\to\infty}f_n(x).
                     \end{equation*}
              \colorbox{cyan}{Definition 3.7.3} 
              (Weierstrass 1841) 
                     Die Folge \\
                     \textcolor{NavyBlue}{$f_n:D\longrightarrow\R$} 
                     \textbf{konvergiert gleichmässig} in 
                     \textcolor{NavyBlue}{$D$} gegen 
                     \textcolor{NavyBlue}{$f:D\longrightarrow\R$}, 
                     wenn gilt: 
                     \textcolor{NavyBlue}{
                     $\forall\varepsilon>0\,\exists N\geqslant1$}, 
                     so dass: 
                     \begin{equation*}
                            \forall n\geqslant N,\,
                            \forall x\in D:\enspace 
                            |f_n(x)-f(x)|<\varepsilon.
                     \end{equation*}
              \colorbox{magenta}{Satz 3.7.4} 
                     Sei \textcolor{NavyBlue}{$D\subseteq\R$} und 
                     \textcolor{NavyBlue}{$f_n:D\longrightarrow\R$} 
                     eine \textbf{Funktionenfolge} bestehend aus 
                     (in $D$) \textbf{stetigen Funktionen}, die 
                     (in $D$) \textbf{gleichmässig gegen eine 
                     Funktion} \textcolor{NavyBlue}{
                     $f:D\longrightarrow\R$} 
                     \textbf{konvergiert}. 
                     Dann ist \textcolor{NavyBlue}{$f$}
                     (in $D$) \textbf{stetig}.\\
              \colorbox{cyan}{Definition 3.7.5} 
                     Eine Funktionenfolge 
                     \textcolor{NavyBlue}{$f_n:D\longrightarrow\R$}
                     ist \textbf{gleichmässig konvergent}, wenn 
                     \textcolor{NavyBlue}{$\forall x\in D$} der 
                     \textbf{Grenzwert} 
                     \textcolor{NavyBlue}{
                     $f(x):=\lim\limits_{n\to\infty}f_n(x)$}
                     \textbf{existiert und die Folge}
                     \textcolor{NavyBlue}{$(f_n)_{n\geqslant0}$} 
                     \textbf{gleichmässig gegen} 
                     \textcolor{NavyBlue}{$f$} 
                     \textbf{konvergiert}.\\
              \colorbox{BurntOrange}{Korollar 3.7.6} 
                     Die Funktionenfolge 
                     \textcolor{NavyBlue}{
                     $f_n:D\longrightarrow\R$}
                     \textbf{konvergiert genau 
                     dann gleichmässig} in $D$, wenn: \\
                     \textcolor{NavyBlue}{
                     $\forall\varepsilon>0\,\exists N\geqslant1$}, 
                     so dass 
                     \textcolor{NavyBlue}{$\forall n,m\geqslant N$} und 
                     \textcolor{NavyBlue}{
                     $\forall x\in D:$}
                     \begin{equation*}
                           |f_n(x)-f_m(x)|<\varepsilon. 
                     \end{equation*}
              \colorbox{BurntOrange}{Korollar 3.7.7} 
                     Sei \textcolor{NavyBlue}{$D\subseteq\R$}. 
                     Wenn 
                     \textcolor{NavyBlue}{$f_n:D\longrightarrow\R$} 
                     eine \textbf{gleichmässig konvergente Folge 
                     stetiger Funktionen} ist, 
                     dann ist die Funktion 
                     \textcolor{NavyBlue}{
                     $f(x):=\lim\limits_{n\to\infty}f_n(x)$} 
                     \textbf{stetig}.\\
              \colorbox{cyan}{Definition 3.7.8} 
                     Die \textbf{Reihe}
                     \textcolor{NavyBlue}{
                     $\sum_{k=0}^\infty f_k(x)$}
                     \textbf{konvergiert gleichmässig} 
                     (in $D$), wenn die durch 
                     \begin{equation*}
                            S_n(x):=\sum_{k=0}^\infty f_k(x)
                     \end{equation*}
                     definierte Funktionenfolge 
                     \textbf{gleichmässig konvergiert}.\\
              \colorbox{magenta}{Satz 3.7.9} 
                     Sei \textcolor{NavyBlue}{$D\subseteq\R$} und 
                     \textcolor{NavyBlue}{
                     $f_n:D\longrightarrow\R$} 
                     eine Folge \textbf{stetiger Funktionen}. 
                     Wir nehmen an, dass \\
                     \textcolor{NavyBlue}{
                     $|f_n(x)|\leqslant c_n
                     \enspace\forall x\in D$}
                     und, dass 
                     \textcolor{NavyBlue}{
                     $\sum_{n=0}^\infty c_n$}
                     \textbf{konvergiert}. 
                     Dann \textbf{konvergiert} die Reihe 
                     \begin{equation*}
                            \sum_{n=0}^\infty f_n(x)
                     \end{equation*}
                     \textbf{gleichmässig} 
                     in $D$ und deren Grenzwert
                     \begin{equation*}
                            f(x):=\sum_{n=0}^\infty f_n(x)
                     \end{equation*}
                     \textbf{ist eine in $D$ stetige Funktion}.\\

              \colorbox{cyan}{Definition 3.7.10} 
                     Die Potenzreihe 
                     \begin{equation*}
                            \sum_{k=0}^\infty c_kx^k
                     \end{equation*}
                     hat \textbf{positiven Konvergenzradius}, 
                     wenn 
                     \begin{equation*}
                            \limsup\limits_{k\to\infty}\sqrt[k]{|c_k|}
                     \end{equation*}
                     existiert. 
                     Der \textbf{Konvergenzradius} 
                     ist dann definiert als: 
                     \begin{align*}
                            \rho&=+\infty 
                            &\text{ für } 
                            \limsup\limits_{k\to\infty}
                            \sqrt[k]{|c_k|}=0 \\
                            \rho&=\frac{1}{\limsup
                            \limits_{k\to\infty}\sqrt[k]{|c_k|}}
                            &\text{ für }
                            \limsup\limits_{k\to\infty}
                            \sqrt[k]{|c_k|}>0
                     \end{align*}
              \colorbox{magenta}{Satz 3.7.11} 
                     Sei \textcolor{NavyBlue}{
                     $\sum_{k=0}^\infty$}
                     eine \textbf{Potenzreihe} mit 
                     \textbf{positivem Konvergenzradius} 
                     \textcolor{NavyBlue}{$\rho>0$} und 
                     \begin{equation*}
                            f(x):=\sum_{k=0}^\infty 
                            c_kc^k,\,|x|<\rho.
                     \end{equation*}
                     Dann gilt: 
                     \textcolor{NavyBlue}{
                     $\forall0\leqslant r<\rho$} 
                     \textbf{konvergiert} 
                     \begin{equation*}
                         \sum_{k=0}^\infty c_kx^k   
                     \end{equation*}
                     \textbf{gleichmässig} auf 
                     \textcolor{NavyBlue}{$[-r,r]$}, 
                     insbesondere ist 
                     \begin{equation*}
                         f:]-\rho,\rho[\longrightarrow\R   
                     \end{equation*}
                     \textbf{stetig}.
       \subsection{Trigonometrische Funktionen}
              \colorbox{magenta}{Satz 3.8.1} 
                     \textcolor{NavyBlue}{$\sin:\R\longrightarrow\R$} 
                     und 
                     \textcolor{NavyBlue}{$\cos:\R\longrightarrow\R$} 
                     sind \textbf{stetige Funktionen}.\\
              \colorbox{magenta}{Satz 3.8.2} 
                     Sei \textcolor{NavyBlue}{$z\in\C$}\\
                     \colorbox{SkyBlue}{(1)} 
                            \textcolor{NavyBlue}{$\exp iz=\cos(z)+i\sin(z)$} \\
                     \colorbox{SkyBlue}{(2)} 
                            \textcolor{NavyBlue}{
                            $\cos(z)=\cos(-z)$} und 
                            \textcolor{NavyBlue}{
                            $\sin(-z)=-\sin(z)$}\\
                     \colorbox{SkyBlue}{(3)} 
                            \textcolor{NavyBlue}{
                            $\sin(z)=\frac{e^{iz}-e^{-iz}}{2i}$},\,
                            \textcolor{NavyBlue}{
                            $\cos(z)=\frac{e^{iz}+e^{-iz}}{2}$}\\
                     \colorbox{SkyBlue}{(4)} 
                            \textcolor{NavyBlue}{
                            $\sin(z+w)
                            =\sin(z)\cos(w)+\cos(z)\sin(w)$}\\
                     \colorbox{SkyBlue}{(5)} 
                            \textcolor{NavyBlue}{
                            $\cos(z+w)
                            =\cos(z)\cos(w)-\sin(z)\sin(w)$}\\ 
                     \colorbox{SkyBlue}{(6)} 
                            \textcolor{NavyBlue}{$\cos^2(z)+\sin^2(z)=1$}\\
              \colorbox{BurntOrange}{Korollar 3.8.3} 
                     \textcolor{NavyBlue}{$\sin(2z)=2\sin(z)\cos(z)$}\\
                     \textcolor{NavyBlue}{$\cos(2z)=\cos^2(z)-\sin^2(z)$}
       \subsection{Die Kreiszahl $\pi$}
              \colorbox{magenta}{Satz 3.9.1} 
                     Die \textbf{Sinusfunktion} hat auf 
                     \textcolor{NavyBlue}{$]0,+\infty[$}
                     \textbf{mindestens eine Nullstelle}. Sei 
                     \begin{equation*}
                            \pi:=\inf\{t>0:\sin t=0\}.
                     \end{equation*}
                     \colorbox{SkyBlue}{(1)} 
                            \textcolor{NavyBlue}{
                            $\sin\pi=0,\,\pi\in]2,4[$} \qquad
                     \colorbox{SkyBlue}{(2)} 
                            \textcolor{NavyBlue}{
                            $\forall x\in]0,\pi[:\sin x>0$}\\
                     \colorbox{SkyBlue}{(3)} 
                            \textcolor{NavyBlue}{
                            $e^{\frac{i\pi}{2}}=i$}\\
              \colorbox{BurntOrange}{Korollar 3.9.2} 
                     \textcolor{NavyBlue}{
                     $x\geqslant\sin x\geqslant 
                     x-\frac{x^3}{3!}\quad
                     \forall0\leqslant x\leqslant\sqrt{6}$}\\
              \colorbox{BurntOrange}{Korollar 3.9.3}
                     Sei \textcolor{NavyBlue}{$x\in\R$}\\
                     \colorbox{SkyBlue}{(1)} 
                            \textcolor{NavyBlue}{
                            $e^{i\pi}=-1,\,e^{2i\pi}=1$}\\ 
                     \colorbox{SkyBlue}{(2)} 
                            \textcolor{NavyBlue}{
                            $\sin(x+\frac{\pi}{2})=\cos(x)$},\,
                            \textcolor{NavyBlue}{
                            $\cos(x+\frac{\pi}{2})=-\sin(x)$}\\ 
                     \colorbox{SkyBlue}{(3)} 
                            \textcolor{NavyBlue}{
                            $\sin(x+\pi)=-\sin(x),\,
                            \sin(x+2\pi)=\sin(x)$}\\
                     \colorbox{SkyBlue}{(4)}
                            \textcolor{NavyBlue}{
                            $\cos(x+\pi)=-\cos(x),\,
                            \cos(x+2\pi)=\cos(x)$}\\
                     \begin{equation*}
                            \text{Sei } k\in\mathbb{Z}
                     \end{equation*}
                     \colorbox{SkyBlue}{(5)}
                            \textbf{Nullstellen von Sinus} 
                            \textcolor{NavyBlue}{
                            $=\{k\cdot\pi:k\in\mathbb{Z}\}$} 
                            \begin{align*}
                                   \sin(x)&>0
                                   &\forall x\in
                                   ]2k\pi,(2k+1)\pi[\\
                                   \sin(x)&<0
                                   &\forall x\in
                                   ](2k+1)\pi,(2k+2)\pi[
                            \end{align*}
                     \colorbox{SkyBlue}{(6)} 
                            \textbf{Nullstellen von Cosinus}
                            \textcolor{NavyBlue}{
                            $=\{\frac{\pi}{2}+k\cdot\pi:
                            k\in\mathbb{Z}\}$}
                            \begin{align*}
                                   \cos(x)&>0
                                   &\forall x\in
                                   ]2k\pi-\frac{\pi}{2},\,
                                   (2k+1)\pi-\frac{\pi}{2}[\\
                                   \cos(x)&<0
                                   &\forall x\in
                                   ](2k+1)\pi-\frac{\pi}{2},\,
                                   (2k+2)\pi-\frac{\pi}{2}[
                            \end{align*}
       \subsection{Grenzwerte von Funktionen}
              \colorbox{cyan}{Definition 3.10.1}
                     \textcolor{NavyBlue}{$x_0\in\R$} 
                     ist ein \textbf{Häufungspunkt} 
                     der Menge \textcolor{NavyBlue}{$D$}, 
                     wenn \textcolor{NavyBlue}{
                     $\forall\delta>0$}:
                     \begin{equation*}
                            (]x_0-\delta,x_0+\delta[)
                            \setminus\{x_0\}\cap D\neq\emptyset  
                     \end{equation*}
              \colorbox{cyan}{Definition 3.10.3}
                     Sei \textcolor{NavyBlue}{
                     $f:D\longrightarrow\R,\,x_0\in\R$}
                     ein \textbf{Häufungspunkt} von 
                     \textcolor{NavyBlue}{$D$}.
                     Dann ist \textcolor{NavyBlue}{$A\in\R$} 
                     der Grenzwert von 
                     \textcolor{NavyBlue}{$f(x)$} für 
                     \textcolor{NavyBlue}{$x\to x_0$}, 
                     bezeichnet mit 
                     \textcolor{NavyBlue}{
                     $\lim\limits_{x\to x_0}f(x)=A$}, 
                     wenn \textcolor{NavyBlue}{
                     $\forall\varepsilon>0\,\exists\delta>0$}, 
                     so dass 
                     \textcolor{NavyBlue}{
                     $\forall x\in 
                     D\cap(]x_0-\delta,x_0+\delta[
                     \setminus\{x_0\})$}:
                     \begin{equation*}
                            |f(x)-A|<\varepsilon
                     \end{equation*}
              \colorbox{green}{Bemerkung 3.10.4} 
                     \colorbox{SkyBlue}{(1)} Sei 
                     \textcolor{NavyBlue}{
                            $f:D\longrightarrow\R$} und 
                            \textcolor{NavyBlue}{$x_0$} ein 
                            \textbf{Häufungspunkt} von $D$. 
                            Dann gilt 
                            \textcolor{NavyBlue}{
                            $\lim\limits_{x\to x_0}f(x)=A$} 
                            \textbf{genau, dann wenn} für 
                            jede Folge 
                            \textcolor{NavyBlue}{
                            $(a_n)_{n\geqslant1}$} 
                            in \textcolor{NavyBlue}{
                            $D\setminus\{x_0\}$} mit 
                            \textcolor{NavyBlue}{
                            $\lim\limits_{n\to\infty}a_n=x_0$} 
                            folgendes gilt: 
                            \textcolor{NavyBlue}{
                            $\lim\limits_{n\to\infty}f(a_n)=A$}.\\ 
                     \colorbox{SkyBlue}{(2)}
                            Sei \textcolor{NavyBlue}{$x_0\in D$}. 
                            Dann ist \textcolor{NavyBlue}{$f$} 
                            \textbf{genau dann stetig, wenn} 
                            \textcolor{NavyBlue}{
                            $\lim\limits_{x\to x_0}f(x)=f(x_0)$}.\\
                     \colorbox{SkyBlue}{(3)} 
                            Mittels \textbf{(1)} zeigt man leicht, 
                            dass wenn \\
                            \textcolor{NavyBlue}{
                            $f,g:D\longrightarrow\R$} 
                            und 
                            \textcolor{NavyBlue}{
                            $\lim\limits_{x\to x_)}f(x),$\,
                            $\lim\limits_{x\to x_0}g(x)$}
                            existieren: 
                            \begin{align*}
                                   \lim\limits_{x\to x_0}(f+g)(x)
                                   &=\lim\limits_{x\to x_0}f(x)
                                   +\lim\limits_{x\to x_0}g(x)\\
                                   \lim\limits_{x\to x_0}(f\cdot g)(x)
                                   &=\lim\limits_{x\to x_0}f(x)
                                   \cdot\lim\limits_{x\to x_0}g(x)
                            \end{align*}
                            folgen.\\
                     \colorbox{SkyBlue}{(4)} 
                            Seien \textcolor{NavyBlue}{
                            $f,g:D\longrightarrow\R$} mit 
                            \textcolor{NavyBlue}{
                            $f\leqslant g$}. Dann folgt 
                            \textcolor{NavyBlue}{
                            $\lim\limits_{x\to x_0}f(x)
                            \leqslant\lim\limits_{x\to x_0}g(x)$}
                            falls beide Grenzwerte existieren.\\
                     \colorbox{SkyBlue}{(5)} 
                            Wenn \textcolor{NavyBlue}{
                            $g_1\leqslant f\leqslant g_2$}und 
                            \textcolor{NavyBlue}{
                            $\lim\limits_{x\to x_0}g_1(x)
                            =\lim\limits_{x\to x_0}g_2(x)$}, 
                            so existiert 
                            \textcolor{NavyBlue}{
                            $l:=\lim\limits_{x\to x_0}f(x)$} 
                            und \textcolor{NavyBlue}{
                            $l=\lim\limits_{x\to x_0}g_1(x)$}\\
              \colorbox{magenta}{Satz 3.10.6} 
                     Seien \textcolor{NavyBlue}{
                     $D,E\subseteq\R,\,x_0$} 
                     \textbf{Häufungspunkt} von 
                     \textcolor{NavyBlue}{$D$},\,
                     \textcolor{NavyBlue}{$f:D\longrightarrow E$} 
                     eine \textbf{Funktion}. Wir nehmen an, dass 
                     \textcolor{NavyBlue}{
                     $y_0:=\lim\limits_{x\to x_0}f(x)$} 
                     existiert und 
                     \textcolor{NavyBlue}{$y_0\in E$}. Wenn 
                     \textcolor{NavyBlue}{
                     $g:E\longrightarrow\R$} in 
                     \textcolor{NavyBlue}{$y_0$} 
                     \textbf{stetig} ist, folgt: 
                     \begin{equation*}
                           \lim\limits_{x\to x_0}g(f(x))
                           =g(y_0). 
                     \end{equation*}

\section{Differenzierbare Funktionen}
       \subsection{Die Ableitung}
              \colorbox{cyan}{Definition 4.1.1} 
                     $f$ ist in $x_0$ \textbf{differenzierbar}, 
                     wenn der Grenzwert \textcolor{NavyBlue}{
                     $\lim\limits_{x\to x_0}\frac{f(x)-f(x-0)}{x-x_0}$}
                     existiert. Ist dies der Fall, wird 
                     der Grenzwert mit \textcolor{NavyBlue}{$f'(x_0)$} bezeichnet.\\
              \colorbox{green}{Bemerkung 4.1.2} 
                     Es ist oft von Vorteil in der Definition von 
                     \textcolor{NavyBlue}{$f'(x_0)$},\,
                     \textcolor{NavyBlue}{$x=x_0+h$} zu setzen, so dass: 
                     \textcolor{NavyBlue}{
                     $f'(x_0)=\lim\limits_{h\to 0}\frac{f(x_0+h)-f(x_0)}{h}$}\\
              \colorbox{magenta}{Satz 4.1.3} (Weierstrass 1861) Sei 
                     \textcolor{NavyBlue}{$f:D\longrightarrow\R,\,x_0\in D$}
                     \textbf{Häufungspunkt} von \textcolor{NavyBlue}{$D$}. 
                     Folgende Aussagen sind 
                     äquivalent: \\
                     \colorbox{SkyBlue}{(1)} \textcolor{NavyBlue}{$f$} ist in 
                            \textcolor{NavyBlue}{$x_0$} 
                            \textbf{differenzierbar} \\
                     \colorbox{SkyBlue}{(2)} Es gibt 
                            \textcolor{NavyBlue}{$c\in\R$} und 
                            \textcolor{NavyBlue}{$r:D\longrightarrow\R$} mit: \\
                            \colorbox{SeaGreen}{2.1} 
                                   \textcolor{NavyBlue}{
                                   $f(x)=f(x_0)+c(x-x_0)+r(x)(x-x_0)$} \\
                            \colorbox{SeaGreen}{2.2} 
                                   \textcolor{NavyBlue}{$r(x_0)=0$} und 
                                   \textcolor{NavyBlue}{$r$} ist \textbf{stetig}
                                   in \textcolor{NavyBlue}{$x_0$}\\
                     Wenn dies zutrifft, ist \textcolor{NavyBlue}{$c=f'(x_0)$} 
                     \textbf{eindeutig} bestimmt.\\
              \colorbox{magenta}{Satz 4.1.4} Eine Funktion 
                     \textcolor{NavyBlue}{$f:D\longrightarrow\R$} ist in 
                     $x_0$ \textbf{genau dann differenzierbar, wenn} es eine 
                     Funktion \textcolor{NavyBlue}{$\phi:D\longrightarrow\R$}
                     gibt, die in \textcolor{NavyBlue}{$x_0$} \textbf{stetig} ist 
                     und \textcolor{NavyBlue}{$f(x)=f(x_0)+\phi(x)(x-x_0)\quad\forall x\in D$}. 
                     In diesem Fall gilt: \textcolor{NavyBlue}{$\phi(x)=f'(x)$}.\\
              \colorbox{BurntOrange}{Korollar 4.1.5} 
                     Sei \textcolor{NavyBlue}{$f:D\longrightarrow\R$}
                     und \textcolor{NavyBlue}{$x_0$} ein \textbf{Häufungspunkt} von 
                     \textcolor{NavyBlue}{$D$}. 
                     Wenn \textcolor{NavyBlue}{$f$} in \textcolor{NavyBlue}{$x_0$} 
                     \textbf{differenzierbar} ist, ist \textcolor{NavyBlue}{$f$} in 
                     \textcolor{NavyBlue}{$x_0$} 
                     \textbf{stetig}.\\
              \colorbox{cyan}{Definition 4.1.7} 
                     \textcolor{NavyBlue}{$f:D\longrightarrow\R$} 
                     ist in \textcolor{NavyBlue}{$D$} \textbf{differenzierbar}, 
                     wenn für \textbf{jeden Häufungspunkt} $x_0\in D,$
                     \textcolor{NavyBlue}{$f$} in \textcolor{NavyBlue}{$x_0$} 
                     \textbf{differenzierbar} ist.\\
              \colorbox{magenta}{Satz 4.1.9} 
                     Sei \textcolor{NavyBlue}{$D\subseteq\R$},\,
                     \textcolor{NavyBlue}{$x_0\in D$} 
                     ein \textbf{Häufungspunkt} von \textcolor{NavyBlue}{$D$} 
                     und \textcolor{NavyBlue}{$f,g:d\longrightarrow\R$} in 
                     \textcolor{NavyBlue}{$x_0$} \textbf{differenzierbar}. 
                     Dann gelten:\\
                     \colorbox{SkyBlue}{(1)} \textcolor{NavyBlue}{$f+g$}
                            ist in \textcolor{NavyBlue}{$x_0$} 
                            \textbf{differenzierbar} und 
                            \textcolor{NavyBlue}{$(f+g)'(x_0)=f'(x_0)+g'(x_0)$}\\
                     \colorbox{SkyBlue}{(2)} 
                            \textcolor{NavyBlue}{$f\cdot g$} 
                            ist in \textcolor{NavyBlue}{$x_0$} 
                            \textbf{differenzierbar} und 
                            \textcolor{NavyBlue}{
                            $(f\cdot g)'(x_0)=f'(x_0)g(x_0)+f(x_0)g'(x_0)$}\\
                     \colorbox{SkyBlue}{(3)} Wenn 
                            \textcolor{NavyBlue}{$g(x_0)\neq0$} ist, ist 
                            \textcolor{NavyBlue}{$\frac{f}{g}$} in 
                            \textcolor{NavyBlue}{$x_0$} \textbf{differenzierbar} und 
                            \textcolor{NavyBlue}{
                            $(\frac{f}{g})'(x_0)
                            =\frac{f'(x_0)g(x_0)-f(x_0)g'(x_0)}{g^2(x_0)}$} \\
              \colorbox{magenta}{Satz 4.1.11}
                     Seien \textcolor{NavyBlue}{$D,E\subseteq\R$}, 
                     sei \textcolor{NavyBlue}{$x_0\in D$}
                     ein \textbf{Häufungspunkt}, sei 
                     \textcolor{NavyBlue}{$f:D\longrightarrow E$}
                     eine in \textcolor{NavyBlue}{$x_0$}
                     \textbf{differenzierbare Funktion}, so dass 
                     \textcolor{NavyBlue}{$y_0:=f(x_0)$} ein 
                     \textbf{Häufungspunkt} von \textcolor{NavyBlue}{$E$} ist und 
                     sei \textcolor{NavyBlue}{$g:E\longrightarrow\R$} eine in 
                     \textcolor{NavyBlue}{$y_0$} 
                     \textbf{differenzierbare Funktion}. Dann ist 
                     \textcolor{NavyBlue}{$g\circ f:D\longrightarrow\R$} in 
                     \textcolor{NavyBlue}{$x_0$} \textbf{differenzierbar} und 
                     \textcolor{NavyBlue}{$(g\circ f)'(x_0)=g'(f(x_0))f'(x_0)$}\\
              \colorbox{BurntOrange}{Korollar 4.1.12}
                     Sei \textcolor{NavyBlue}{$f:D\longrightarrow E$} eine 
                     \textbf{bijektive Funktion}, sei 
                     \textcolor{NavyBlue}{$x_0\in D$} ein \textbf{Häufungspunkt}. 
                     Wir nehmen an \textcolor{NavyBlue}{$f$} ist 
                     in \textcolor{NavyBlue}{$x_0$} \textbf{differenzierbar} und 
                     \textcolor{NavyBlue}{$f'(x_0)\neq0$}. Zudem nehmen 
                     wir an \textcolor{NavyBlue}{$f^{-1}$} ist in \textcolor{NavyBlue}{$y_0=f(x_0)$}
                     \textbf{stetig}. Dann ist \textcolor{NavyBlue}{$y_0$} ein 
                     \textbf{Häufungspunkt} von \textcolor{NavyBlue}{$E$}, 
                     \textcolor{NavyBlue}{$f^{-1}$} ist in \textcolor{NavyBlue}{$y_0$} 
                     \textbf{differenzierbar} 
                     und \textcolor{NavyBlue}{
                     $(f^{-1})'(y_0)=\frac{1}{f'(x_0)}$}
       \subsection{Zentrale Sätze über die Ableitung}
              \colorbox{cyan}{Definition 4.2.1} 
                     Sei \textcolor{NavyBlue}{$f:D\longrightarrow\R$},\,
                     \textcolor{NavyBlue}{$D\subseteq\R$}
                     und \textcolor{NavyBlue}{$x_0\in D$}. \\
                     \colorbox{SkyBlue}{(1)}[\colorbox{SkyBlue}{(2)}] 
                            \textcolor{NavyBlue}{$f$} besitzt ein 
                            \textbf{lokales Maximum [Minimum]} in 
                            \textcolor{NavyBlue}{$x_0$}, wenn 
                            \textcolor{NavyBlue}{$\exists\delta>0$} mit: 
                            \textcolor{NavyBlue}{
                            $f(x)\leqslant[\geqslant] f(x_0)\quad
                            \forall x\in]x_0-\delta,x_0+\delta[\cap D$}\\
                     \colorbox{SkyBlue}{(3)} 
                            $f$ besitzt ein \textbf{lokales Extremum} 
                            in \textcolor{NavyBlue}{$x_0$}, wenn es ein 
                            \textbf{lokales Minimum oder Maximum} ist. \\
              \colorbox{magenta}{Satz 4.2.2} 
                     Sei \textcolor{NavyBlue}{
                     $f:]a,b[\longrightarrow\R,\,x_0\in]a,b[$}. 
                     Wir nehmen an, \textcolor{NavyBlue}{$f$} ist in 
                     \textcolor{NavyBlue}{$x_0$} \textbf{differenzierbar}.\\
                     \colorbox{SkyBlue}{(1)} Wenn \textcolor{NavyBlue}{$f'(x_0)>0$} 
                            ist, \textcolor{NavyBlue}{$\exists\delta>0$} mit: 
                            \textcolor{NavyBlue}{
                            $f(x)>f(x_0)\quad\forall x\in]x_0,x_0+\delta[$}
                            \textcolor{NavyBlue}{
                            $f(x)<f(x_0)\quad\forall x\in]x_0-\delta,x_0[$}\\
                     \colorbox{SkyBlue}{(2)} Wenn 
                            \textcolor{NavyBlue}{$f'(x_0)<0$} ist, 
                            \textcolor{NavyBlue}{$\exists\delta>0$} mit: 
                            \textcolor{NavyBlue}{
                            $f(x)<f(x_0)\quad\forall x\in]x_0,x_0+\delta[$}
                            \textcolor{NavyBlue}{
                            $f(x)>f(x_0)\quad\forall x\in]x_0-\delta,x_0[$}\\ 
                     \colorbox{SkyBlue}{(3)} Wenn \textcolor{NavyBlue}{$f$} in 
                            \textcolor{NavyBlue}{$x_0$} ein 
                            \textbf{lokales Extremum} 
                            besitzt, folgt \textcolor{NavyBlue}{$f'(x_0)=0$}.\\
              \colorbox{magenta}{Satz 4.2.3} 
              (Rolle 1690) 
                     Sei \textcolor{NavyBlue}{$f:[a,b]\longrightarrow\R$}
                     stetig und in \textcolor{NavyBlue}{$]a,b[$} 
                     \textbf{differenzierbar}. 
                     Wenn \textcolor{NavyBlue}{$f(a)=f(b)$}, so gibt es 
                     \textcolor{NavyBlue}{$\xi\in]a,b[$} mit: 
                     \textcolor{NavyBlue}{$f'(\xi)=0$}.\\
              \colorbox{magenta}{Satz 4.2.4}
              (Lagrange 1797) 
                     Sei \textcolor{NavyBlue}{$f\longrightarrow\R$}
                     \textbf{stetig} und in \textcolor{NavyBlue}{$]a,b[$} 
                     \textbf{differenzierbar}. 
                     Es gibt \textcolor{NavyBlue}{$\xi\in]a,b[$} mit: 
                     \textcolor{NavyBlue}{$f(b)-f(a)=f'(\xi)(b-a)$}.\\
              \colorbox{BurntOrange}{Korollar 4.2.5} 
                     Seien \textcolor{NavyBlue}{$f,g:[a,b]\longrightarrow\R$}
                     \textbf{stetig} und in \textcolor{NavyBlue}{$]a,b[$} 
                     \textbf{differenzierbar}.\\
                     \colorbox{SkyBlue}{(1)} Wenn 
                            \textcolor{NavyBlue}{$f'(\xi)=0\quad\forall\xi\in]a,b[$} ist, 
                            ist \textcolor{NavyBlue}{$f$} \textbf{konstant}.\\
                     \colorbox{SkyBlue}{(2)} Wenn 
                            \textcolor{NavyBlue}{$f'(\xi)=g'(\xi)\quad\forall\xi\in]a,b[$} 
                            ist, gibt es \textcolor{NavyBlue}{$c\in\R$} mit: 
                            \textcolor{NavyBlue}{
                            $f(x)=g(x)+c\quad\forall x\in[a,b]$}.\\
                     \colorbox{SkyBlue}{(3)}[\colorbox{SkyBlue}{(4)}] 
                            Wenn \textcolor{NavyBlue}{
                            $f'(\xi)\geqslant[>]0\quad\forall\xi\in]a,b[$} ist, 
                            ist \textcolor{NavyBlue}{$f$} auf 
                            \textcolor{NavyBlue}{$[a,b]$} 
                            \textbf{[strikt] monoton wachsend}.\\
                     \colorbox{SkyBlue}{(5)}[\colorbox{SkyBlue}{(6)}] 
                            Wenn \textcolor{NavyBlue}{
                            $f'(\xi)\leqslant[<]0\quad\forall\xi\in]a,b[$} ist, 
                            ist \textcolor{NavyBlue}{$f$} auf 
                            \textcolor{NavyBlue}{$[a,b]$} 
                            \textbf{[strikt] monoton fallend}.\\
                     \colorbox{SkyBlue}{(7)} Wenn es 
                            \textcolor{NavyBlue}{$M\geqslant0$} gibt, mit: 
                            \textcolor{NavyBlue}{
                            $|f'(\xi)|\leqslant M\quad\forall\xi\in]a,b[$}, 
                            folgt \textcolor{NavyBlue}{ 
                            $\forall x_1,x_2\in[a,b]:\enspace
                            |f(x_1)-f(x_2)|\leqslant M|x_1-x_2|$}. \\
              \colorbox{Dandelion}{Beispiel 4.2.6}
                     \colorbox{SkyBlue}{(1) $\arcsin$} 
                            Da $\sin'=\cos$ und 
                            $\cos(x)>0\enspace
                            \forall x\in]-\frac{\pi}{2},\frac{\pi}{2}[$
                            folgt aus \textbf{Korollar 4.2.5 (4)}, dass die 
                            Sinusfunktion auf
                            $[-\frac{\pi}{2},\frac{\pi}{2}]$ strikt 
                            monoton wachsend
                            ist. Also ist 
                            $\sin:[-\frac{\pi}{2},\frac{\pi}{2}]\longrightarrow[-1,1]$ 
                            bijektiv. Wir definieren
                            $\arcsin:[-1,1]\longrightarrow[-\frac{\pi}{2},\frac{\pi}{2}]$
                            als die Umkehrfunktion von $\sin$. Nach 
                            \textbf{Korollar 4.1.12} 
                            ist sie auf $]-1,1[$ differenzierbar und für
                            $y=\sin x,\enspace x\in]-\frac{\pi}{2},\frac{\pi}{2}[$
                            folgt nach \textbf{4.1.12}:
                            $\arcsin'(y)=\frac{1}{\sin'(x)}=\frac{1}{\cos x}$.
                            Wir verwenden nun: $y^2=\sin^2x=1-\cos^2x$
                            woraus mit $\cos c>0$ folgt:
                            $\cos x=\sqrt{1-y^2}$. Wir erhalten also 
                            $\forall y\in ]-1,1[$
                            $\arcsin'(y)=\frac{1}{\sqrt{1-y^2}}$.\\
                     \colorbox{SkyBlue}{(2) $\arccos$}
                            Eine analoge Diskussion, wie in (1) zeigt, dass
                            $\cos:[0,\pi]\longrightarrow[-1,]$ strikt 
                            monoton fallend ist
                            und $[0,\pi]$ auf $[-1,1]$ bijektiv abbildet. Sei:
                            $\arccos:[-1,1]\longrightarrow[0,\pi]$ die 
                            Umkehrfunktion. Sie ist auf $]-1,1[$ 
                            differenzierbar und
                            $\arccos'(y)=\frac{-1}{\sqrt{1-y^2}}\enspace
                            \forall y\in]-1,1[$.\\
                     \colorbox{SkyBlue}{(3) $\arctan$}
                            Für $x\notin\frac{\pi}{2}+\pi\cdot\mathbb{Z}$
                            hatten wir die Tangensfunktion definiert:
                            $\tan x=\frac{\sin x}{\cos x}$ und deren Ableitung 
                            berechnet:
                            $\tan'x=\frac{1}{cos^2x}$. Also ist $\tan$ auf
                            $]-\frac{\pi}{2},\frac{\pi}{2}[$ streng monoton wachsend mit
                            $\lim\limits_{x\to\frac{\pi}{2}^{-}}\tan x=+\infty$,
                            $\lim\limits_{x\to\frac{\pi}{2}^{+}}\tan x=-\infty$.
                            Also ist 
                            $\tan:]-\frac{\pi}{2},\frac{\pi}{2}[
                            \longrightarrow]-\infty,\infty[$ bijektiv. Sei
                            $\arctan:]-\infty,\infty[\longrightarrow
                            ]-\frac{\pi}{2},\frac{\pi}{2}[$ die 
                            Umkehrfunktion.
                            Dann ist $\arctan$ differenzierbar und für 
                            $y=\tan x$: $\arctan'(y)=\cos^x=\frac{1}{1+y^2}$.\\
              \colorbox{magenta}{Satz 4.2.9} 
              (Cauchy)
                     Seien 
                     \textcolor{NavyBlue}{$f,g:[a,b]\longrightarrow\R$}
                     stetig und in \textcolor{NavyBlue}{$]a,b[$} 
                     \textbf{differenzierbar}.
                     Es gibt \textcolor{NavyBlue}{$\xi\in]a,b[$} mit: 
                     \textcolor{NavyBlue}{$g'(\xi)(f(b)-f(a))=f'(\xi)(g(b)-g(a))$}. 
                     Wenn \textcolor{NavyBlue}{$g'(x)\neq0\quad\forall x\in]a,b[$} 
                     ist, folgt: \textcolor{NavyBlue}{$g(a)\neq g(b)$} und 
                     \textcolor{NavyBlue}{
                     $\frac{f(b)-f(a)}{g(b)-g(a)}=\frac{f'(\xi)}{g'(\xi)}$}\\ 
              \colorbox{magenta}{Satz 4.2.10} 
              (\textbf{Bernoulli} 1691/92, \textbf{de l'Hôpital} 1696) 
                     Seien \textcolor{NavyBlue}{$f,g:]a,b[\longrightarrow\R$} 
                     \textbf{differenzierbar} mit 
                     \textcolor{NavyBlue}{$g'(x)\neq0\quad\forall x\in]a,b[$}. 
                     Wenn \textcolor{NavyBlue}{
                     $\lim\limits_{x\to b^{-}}f(x)=0,\,\lim\limits_{x\to b^{-}}g(x)=0$}
                     und \textcolor{NavyBlue}{
                     $\lim\limits_{x\to b^{-}}\frac{f'(x)}{g'(x)}=:\lambda$}
                     existiert, folgt: \textcolor{NavyBlue}{
                     $\lim\limits_{x\to b^{-}}\frac{f(x)}{g(x)}
                     =\lim\limits_{x\to b^{-}}\frac{f'(x)}{g'(x)}$}.\\
              \colorbox{green}{Bemerkung 4.2.11} 
                     Der Satz \textbf{gilt auch} wenn: 
                     \textcolor{NavyBlue}{
                     $b=+\infty\qquad\lambda=+\infty\qquad x\to a^{+}$}\\
              \colorbox{cyan}{Definition 4.2.13} 
                     \colorbox{SkyBlue}{(1)} 
                            \textcolor{NavyBlue}{$f$}
                            ist \textbf{konvex} (auf \textcolor{NavyBlue}{$I$}), wenn 
                            \textcolor{NavyBlue}{$\forall x,y\in I,\,x\leqslant y$ 
                            und $\lambda\in[0,1]$}
                            folgendes gilt: \textcolor{NavyBlue}{
                            $f(\lambda x+(1-\lambda)y)
                            \leqslant\lambda f(x)+(1-\lambda)f(y)$}\\
                     \colorbox{SkyBlue}{(2)} 
                            \textcolor{NavyBlue}{$f$} ist 
                            \textbf{streng konvex}, 
                            wenn \textcolor{NavyBlue}{$\forall x,y\in I,\,x<y$}
                            und \textcolor{NavyBlue}{$\lambda\in]0,1[$} 
                            folgendes gilt: 
                            \textcolor{NavyBlue}{
                            $f(\lambda x+(1-\lambda)y)
                            <\lambda f(x)+(1-\lambda)f(y)$}\\ 
              \colorbox{green}{Bemerkung 4.2.14} 
                     Sei \textcolor{NavyBlue}{$f:I\longrightarrow\R$} 
                     \textbf{konvex}. Ein einfacher Induktionsbeweis zeigt, dass 
                     \textcolor{NavyBlue}{
                     $\forall n\geqslant1,\,\{x_1,...,x_n\}\subseteq I$} 
                     und \textcolor{NavyBlue}{$\lambda_1,...\lambda_n$} in 
                     \textcolor{NavyBlue}{$[0,1]$} mit 
                     \textcolor{NavyBlue}{$\sum_{i=1}^n\lambda_i=1$}
                     folgendes gilt: \textcolor{NavyBlue}{
                     $f(\sum_{i=1}^n\lambda_i x_i)
                     \leqslant\sum_{i=1}^n\lambda_if(x_i)$}\\
              \colorbox{yellow}{Lemma 4.2.15} 
                     Sei \textcolor{NavyBlue}{$f:I\longrightarrow\R$} 
                     eine beliebige Funktion. Die Funktion \textcolor{NavyBlue}{$f$}
                     ist \textbf{genau dann konvex, wenn} 
                     \textcolor{NavyBlue}{$\forall x_0<x<x_1$} in 
                     \textcolor{NavyBlue}{$I$}
                     folgendes gilt: \textcolor{NavyBlue}{
                     $\frac{f(x)-f(x_0)}{x-x_0}\leqslant\frac{f(x_1)-f(x)}{x_1-x}$}\\ 
              \colorbox{magenta}{Satz 4.2.16} 
                     Sei \textcolor{NavyBlue}{$f:]a,b[\longrightarrow\R$}
                     \textbf{differenzierbar}. 
                     \textcolor{NavyBlue}{$f$} ist 
                     \textbf{genau dann (streng) konvex, wenn} 
                     \textcolor{NavyBlue}{$f'$} \textbf{(streng) monoton
                     wachsend} ist.\\
              \colorbox{BurntOrange}{Korollar 4.2.17} 
                     Sei \textcolor{NavyBlue}{$f:]a,b[\longrightarrow\R$}
                     zweimal differenzierbar in \textcolor{NavyBlue}{$]a,b[$}. 
                     \textcolor{NavyBlue}{$f$} ist \textbf{(streng) konvex}, 
                     wenn \textcolor{NavyBlue}{$f''\geqslant0$} (bzw. 
                     $\textcolor{NavyBlue}{f''>0}$) auf 
                     \textcolor{NavyBlue}{$]a,b[$}.
       \subsection{Höhere Ableitungen}
              \colorbox{cyan}{Definition 4.3.1} 
                     \colorbox{SkyBlue}{(1)} Für $n\geqslant2$ ist 
                            $f\,n$\textbf{-mal differenzierbar in} $D$, wenn 
                            \textcolor{NavyBlue}{$f^{(n-1)}$} in \textcolor{NavyBlue}{$D$}
                            \textbf{differenzierbar} ist. 
                            Dann ist \textcolor{NavyBlue}{$f^{(n)}:=(f^{(n-1)})'$} 
                            und nennt sich die $n$\textbf{-te Ableitung} von 
                            \textcolor{NavyBlue}{$f$}\\
                     \colorbox{SkyBlue}{(2)} \textcolor{NavyBlue}{$f$} 
                            ist $n$\textbf{-mal stetig differenzierbar} in 
                            \textcolor{NavyBlue}{$D$}, 
                            wenn \textcolor{NavyBlue}{$f$} 
                            \textbf{$n$-mal differenzierbar} in
                            \textcolor{NavyBlue}{$D$}
                            \& \textcolor{NavyBlue}{$f^{(n)}$} 
                            in \textcolor{NavyBlue}{$D$} \textbf{stetig} ist.\\
                     \colorbox{SkyBlue}{(3)} Die Funktion \textcolor{NavyBlue}{$f$} 
                            ist in \textcolor{NavyBlue}{$D$} \textbf{glatt}, wenn sie 
                            \textcolor{NavyBlue}{$\forall n\geqslant1$}\,
                            \textbf{$n$-mal differenzierbar} ist.\\
              \colorbox{green}{Bemerkung 4.3.2} 
                     Es folgt aus \textbf{Korollar 4.1.5}, dass für 
                     \textcolor{NavyBlue}{$n\geqslant1$} eine 
                     \textbf{$n$-mal differenzierbare}
                     Funktion \textbf{$(n-1)$-mal 
                     stetig differenzierbar} ist.\\
              \colorbox{magenta}{Satz 4.3.3} 
              (\textbf{analog zu Satz 4.1.9}) 
                     Sei \textcolor{NavyBlue}{$D\subseteq\R$} wie in 
                     \textbf{Definition 4.3.1}, 
                     \textcolor{NavyBlue}{$n\geqslant1$} und 
                     \textcolor{NavyBlue}{$f,g:D\longrightarrow\R$}\,
                     \textbf{$n$-mal 
                     differenzierbar} in \textcolor{NavyBlue}{$D$}.\\
                     \colorbox{SkyBlue}{(1)} \textcolor{NavyBlue}{$f+g$} 
                            ist $n$\textbf{-mal differenzierbar} und 
                            \textcolor{NavyBlue}{$(f+g)^{(n)}=f^{(n)}+g^{(n)}$}\\
                     \colorbox{SkyBlue}{(2)} \textcolor{NavyBlue}{$f\cdot g$}
                            ist $n$\textbf{-mal differenzierbar} 
                            und \textcolor{NavyBlue}{
                            $(f\cdot g)^{(n)}
                            =\sum_{k=0}^n\binom{n}{k}f^{(k)}g^{(n-k)}$}.\\
              \colorbox{magenta}{Satz 4.3.5} 
                     Sei \textcolor{NavyBlue}{$D\subseteq\R$} wie in 
                     \textbf{Definition 4.3.1}, 
                     \textcolor{NavyBlue}{$n\geqslant1$} und 
                     \textcolor{NavyBlue}{$f,g:D\longrightarrow\R$}\,
                     \textbf{$n$-mal differenzierbar} in 
                     \textcolor{NavyBlue}{$D$}. Wenn 
                     \textcolor{NavyBlue}{$g(x)\neq0\quad\forall x\in D$} ist, 
                     ist \textcolor{NavyBlue}{$\frac{f}{g}$} 
                     in \textcolor{NavyBlue}{$D$} 
                     $n$\textbf{-mal differenzierbar}. \\
              \colorbox{magenta}{Satz 4.3.6} 
                     Seien \textcolor{NavyBlue}{$E,D\subseteq\R$} Teilmengen 
                     für die \textbf{jeder Punkt Häufungspunkt} ist. Seien 
                     \textcolor{NavyBlue}{$f:D\longrightarrow E$},
                     \textcolor{NavyBlue}{$g:E\longrightarrow\R$} 
                     \textbf{$n$-mal differenzierba}r.
                     Dann ist \textcolor{NavyBlue}{$f\circ g$} 
                     $n$\textbf{-mal differenzierbar} und 
                     \textcolor{NavyBlue}{
                     $(g\circ f)^{(n)}(x)
                     =\sum_{k=1}^nA_{n,k}(x)(g^{(k)}\circ f)(x)$}
                     wobei \textcolor{NavyBlue}{$A_{n,k}$} 
                     ein Polynom in den Funktionen 
                     \textcolor{NavyBlue}{$f',f^{(2)},...,f^{n+1-k)}$} ist.
       \subsection{Potenzreihen \& Taylor Approximation}
              \colorbox{magenta}{Satz 4.4.1} 
                     Seien \textcolor{NavyBlue}{$f_n:]a,b[\longrightarrow\R$} 
                     eine Funktionenfolge wobei \textcolor{NavyBlue}{$f_n$} 
                     \textbf{einmal} in 
                     \textcolor{NavyBlue}{$]a,b[\quad\forall n\geqslant1$} 
                     \textbf{stetig differenzierbar} ist.
                     Wir nehmen an, dass sowohl die Folge 
                     \textcolor{NavyBlue}{$(f_n)_{n\geqslant1}$} 
                     wie auch \textcolor{NavyBlue}{$(f'_n)_{n\geqslant1}$} 
                     \textbf{gleichmässig} in \textcolor{NavyBlue}{$]a,b[$} mit 
                     \textcolor{NavyBlue}{$\lim\limits_{n\to\infty}f_n=:f$} 
                     und \textcolor{NavyBlue}{$\lim\limits_{n\to\infty}f'_n=:p$}
                     \textbf{konvergieren}. Dann ist \textcolor{NavyBlue}{$f$}
                     \textbf{stetig differenzierbar} 
                     und \textcolor{NavyBlue}{$f'=p$}. \\
              \colorbox{magenta}{Satz 4.4.2} 
                     Sei \textcolor{NavyBlue}{$\sum_{k=0}^\infty c_kx^k$} 
                     eine \textbf{Potenzreihe} mit 
                     pos. Konvergenzradius \textcolor{NavyBlue}{$\rho>0$}. Dann ist
                     \textcolor{NavyBlue}{$f(x)=\sum_{k=0}^\infty c_k(x-x_0)^k$} 
                     auf \textcolor{NavyBlue}{$]x_0-\rho,x_0+\rho[$} 
                     \textbf{differenzierbar} und 
                     \textcolor{NavyBlue}{
                     $f'(x)=\sum_{k=0}^\infty kc_k(x-x_0)^{k-1}\quad
                     \forall x\in]x_0-\rho,x_0+\rho[$}.\\
              \colorbox{BurntOrange}{Korollar 4.4.3} 
                     Unter der Voraussetzung von \textbf{Satz 4.4.1} ist 
                     \textcolor{NavyBlue}{$f$} auf 
                     \textcolor{NavyBlue}{$]x_0-\rho,x_0+\rho[$} 
                     \textbf{glatt} und \textcolor{NavyBlue}{$f^{(j)}(x)
                     =\sum_{k=j}^\infty c_k\frac{k!}{(k-j)!}(x-x_0)^{k-j}$}. 
                     Insbesondere ist 
                     \textcolor{NavyBlue}{$c_j=\frac{f^{(j)}(x_0)}{j!}$}.\\
              \colorbox{magenta}{Satz 4.4.5} 
                     Sei \textcolor{NavyBlue}{$f:[a,b]\longrightarrow\R$}
                     \textbf{stetig} und in 
                     \textcolor{NavyBlue}{$]a,b[$} 
                     \textbf{$(n+1)$-mal differenzierbar}. 
                     Für jedes \textcolor{NavyBlue}{$a<x\leqslant b$} gibt es 
                     \textcolor{NavyBlue}{$\xi\in]a,x[$} mit: 
                     \textcolor{NavyBlue}{
                     $f(x)=\sum_{k=0}^\infty\frac{f^{(k)}(a)}{k!}(x-a)^k
                     +\frac{f^{(n+1)}(\xi)}{(n+1)!}(x-a)^{n+1}$}. \\
              \colorbox{BurntOrange}{Korollar 4.4.6} 
              (\textbf{Taylor Approximation}) 
                     Sei \textcolor{NavyBlue}{$f:[c,d]\longrightarrow\R$} 
                     \textbf{stetig} und in 
                     \textcolor{NavyBlue}{$]c,d[$} 
                     \textbf{$(n+1)$-mal differenzierbar}. Sei 
                     \textcolor{NavyBlue}{$c<a<d$}. 
                     \textcolor{NavyBlue}{$\forall x\in[c,d]\,\exists\xi$} zwischen 
                     \textcolor{NavyBlue}{$x$} und \textcolor{NavyBlue}{$a$}, 
                     so dass: \textcolor{NavyBlue}{
                     $f(x)=\sum_{k=0}^n\frac{f^{(k)}(a)}{k!}(x-a)^k
                     +\frac{f^{(n+1)}(\xi)}{(n+1)!}(x-a)^{n+1}$}.\\
              \colorbox{BurntOrange}{Korollar 4.4.7} 
                     Sei \textcolor{NavyBlue}{$n\geqslant0,\,a<x_0<b$} und 
                     \textcolor{NavyBlue}{$f:[a,b]\longrightarrow\R$} 
                     in \textcolor{NavyBlue}{$]a,b[$} 
                     \textbf{$(n+1)$-mal stetig differenzierbar}.
                     Annahme: \textcolor{NavyBlue}{
                     $f'(x_0)=f^{(2)}(x_0)=...=f^{(n)}(x_0)=0$}. \\
                     \colorbox{SkyBlue}{(1)} 
                            Wenn \textcolor{NavyBlue}{$n$} 
                            \textbf{gerade} 
                            ist und \textcolor{NavyBlue}{$x_0$} eine 
                            \textbf{lokale Extremalstelle} ist, folgt
                            \textcolor{NavyBlue}{ $f^{(n+1)}(x_0)=0$}. \\
                     \colorbox{SkyBlue}{(2)} Wenn \textcolor{NavyBlue}{$n$} 
                            \textbf{ungerade}
                            ist und 
                            \textcolor{NavyBlue}{$f^{(n+1)}(x_0)>0$} ist, ist 
                            \textcolor{NavyBlue}{$x_0$} 
                            eine \textbf{strikt lokale Minimalstelle}. \\
                     \colorbox{SkyBlue}{(3)} Wenn \textcolor{NavyBlue}{$n$} 
                            \textbf{ungerade} 
                            ist und \textcolor{NavyBlue}{$f^{(n+1)}(x_0)<0$} ist,
                            \textcolor{NavyBlue}{$x_0$} eine 
                            \textbf{strikt lokale Maximalstelle}. \\
              \colorbox{BurntOrange}{Korollar 4.4.8} 
                     Sei \textcolor{NavyBlue}{$f:[a,b]\longrightarrow\R$} 
                     \textbf{stetig} und in \textcolor{NavyBlue}{$]a,b[$}
                     \textbf{zweimal stetig differenzierbar}. Sei 
                     \textcolor{NavyBlue}{$a<x_0<b$}. 
                     Annahme: \textcolor{NavyBlue}{$f'(x_0)=0$}.\\
                     \colorbox{SkyBlue}{(1)} Wenn 
                            \textcolor{NavyBlue}{$f^{(2)}(x_0)>0$} ist, ist 
                            \textcolor{NavyBlue}{$x_0$}
                            eine \textbf{strikt lokale Minimalstelle}. \\
                     \colorbox{SkyBlue}{(2)} Wenn 
                            \textcolor{NavyBlue}{$f^{(2)}(x_0)<0$} ist, ist
                            \textcolor{NavyBlue}{$x_0$}
                            eine \textbf{strikt lokale Maximalstelle}.

\section{Riemann Integral}
       \subsection{Definition und Integrabilitätskriterien}
              \colorbox{cyan}{Definition 5.1.1} 
                     Eine \textbf{Partition} von \textcolor{NavyBlue}{$I$} 
                     ist eine endliche Teilmenge 
                     \textcolor{NavyBlue}{$P\subsetneq[a,b]$} 
                     wobei \textcolor{NavyBlue}{$\{a,b\}\subseteq P$}. 
                     Es gilt: 
                     \textcolor{NavyBlue}{$n:=\operatorname{card} P-1\geqslant1$} 
                     und es gibt \textbf{genau eine Bijektion} 
                     \textcolor{NavyBlue}{
                     $\{0,1,2,...,n\}\longrightarrow P,\,j\mapsto x_j$}
                     mit der Eigenschaft 
                     \textcolor{NavyBlue}{$i<j\Longrightarrow x_i<x_j$}.\\
              Eine \textbf{Partition} \textcolor{NavyBlue}{$P'$ }ist eine 
                     \colorbox{gray}{Verfeinerung} von \textcolor{NavyBlue}{$P$}, 
                     wenn \textcolor{NavyBlue}{$P\subset P'$}. 
                     Offensichtlich ist die \textbf{Vereinigung} 
                     \textcolor{NavyBlue}{$P_1\cup P_2$} \textbf{zweier Partitionen} 
                     wieder \textbf{eine Partition}. Insbesondere 
                     haben \textbf{zwei Partitionen immer} 
                     eine \textbf{gemeinsame Vereinigung}.
              Sei \textcolor{NavyBlue}{$f:[a,b]\longrightarrow\R$} eine 
                     \textbf{beschränkte Funktion}, 
                     das heisst es gibt 
                     \textcolor{NavyBlue}{$M\geqslant0$} mit 
                     \textcolor{NavyBlue}{$|f(x)|\leqslant M\quad\forall x\in[a,b]$}. 
              Sei \textcolor{NavyBlue}{$P=\{x_0,x_1,...,x_n\}$} eine 
                     \textbf{Partition} von 
                     \textcolor{NavyBlue}{$I$}. Insbesondere gilt: 
                     \textcolor{NavyBlue}{$x_0=a<x_1<...<x_n=b$}
              \textbf{Länge des Teilintervalls} 
                     \textcolor{NavyBlue}{$[x_{i-1},x_i]$},\,
                     \textcolor{NavyBlue}{$\delta_i:=x_i-x_{i-1},\,i\geqslant1$}\\
              \colorbox{gray}{Untersumme} 
                     \textcolor{NavyBlue}{$s(f,P):=\sum_{i=1}^nf_i\delta_i,\,
                     f_i=\inf\limits_{x_{i-1}\leqslant x\leqslant x_i}f(x)$}\\ 
              \colorbox{gray}{Obersumme} 
                     \textcolor{NavyBlue}{$S(f,P):=\sum_{i=1}^nF_i\delta_i,\, 
                     F_i=\sup\limits_{x_{i-1}\leqslant x\leqslant x_i}f(x)$}\\
              \colorbox{yellow}{Lemma 5.1.2} 
                     \colorbox{SkyBlue}{(1)} Sei \textcolor{NavyBlue}{$P'$}
                     eine \textbf{Verfeinerung} von \textcolor{NavyBlue}{$P$}. Dann gilt: 
                     \textcolor{NavyBlue}{
                     $s(f,P)\leqslant s(f,P')\leqslant S(f,P')\leqslant S(f,P)$}.\\
                     \colorbox{SkyBlue}{(2)} Für beliebige Partitionen 
                     \textcolor{NavyBlue}{$P_1,P_2$}
                     gilt: \textcolor{NavyBlue}{$s(f,P_1)\leqslant S(f,P_2)$}.\\
              Sei \textcolor{NavyBlue}{$\mathcal{P}(I)$} 
                     die \colorbox{gray}{Menge der Partitionen} 
                     von \textcolor{NavyBlue}{$I$}. Wir definieren: 
                     \textcolor{NavyBlue}{
                     $s(f)=\sup\limits_{P\in\mathcal{P}(I)}s(f,P)$},\quad 
                     \textcolor{NavyBlue}{
                     $S(f)=\inf\limits_{P\in\mathcal{P}(I)}S(f,P)$}. \\
              \colorbox{cyan}{Definition 5.1.3} 
                     Eine \textbf{beschränkte Funktion} 
                     \textcolor{NavyBlue}{$f:[a,b]\longrightarrow\R$} 
                     ist \textbf{(Riemann) integrierbar}, wenn \textcolor{NavyBlue}{$s(f)=S(f)$}. 
                     In diesem Fall 
                     bezeichnen wir den \textbf{gemeinsamen Wert} 
                     von \textcolor{NavyBlue}{$s(f)$} und 
                     \textcolor{NavyBlue}{$S(f)$} mit 
                     \textcolor{NavyBlue}{$\int_a^bf(x)\,dx$}.\\
              \colorbox{magenta}{Satz 5.1.4} 
              Eine \textbf{beschränkte Funktion} ist 
                     genau dann integrierbar, wenn 
                     \textcolor{NavyBlue}{
                     $\forall\varepsilon>0\,\exists P\in\mathcal{P}(I)$}:
                     \textcolor{NavyBlue}{$S(f,P)-s(f,P)<\varepsilon$}.\\
              \colorbox{magenta}{Satz 5.1.8} 
              (Du Bois-Reymond 1875, Darboux 1875) 
                     Eine \textbf{beschränkte Funktion} 
                     \textcolor{NavyBlue}{$f:[a,b]\longrightarrow\R$} 
                     ist \textbf{genau dann 
                     integrierbar, wenn} 
                     \textcolor{NavyBlue}{$\forall\varepsilon>0\,\exists\delta>0$}, 
                     so dass: \textcolor{NavyBlue}{
                     $\forall P\in\mathcal{P}_\delta(I),\,S(f,P)-s(f,P)<\varepsilon$}.
                     Hier bezeichnet 
                     \textcolor{NavyBlue}{$\mathcal{P}_\delta(I)$} die 
                     \textbf{Menge der Partitionen $P$}, für welche 
                     \textcolor{NavyBlue}{
                     $\max\limits_{1\leqslant i\leqslant n}\delta_i
                     \leqslant\delta$}. \\
              \colorbox{BurntOrange}{Korollar 5.1.9} 
                     Die \textbf{beschränkte Funktion} 
                     \textcolor{NavyBlue}{$f:[a,b]\longrightarrow\R$} ist 
                     \textbf{genau dann integrierbar mit}
                     \textcolor{NavyBlue}{$A:=\int_a^bf(x)\,dx$}, 
                     \textbf{wenn}: 
                     \textcolor{NavyBlue}{
                     $\forall\varepsilon>0\enspace\exists\delta>0$}, so dass 
                     \textcolor{NavyBlue}{$\forall P\in\mathcal{P}(I)$} mit 
                     \textcolor{NavyBlue}{$\delta(P)<\delta$}
                     und \textcolor{NavyBlue}{$\xi_1,...,\xi_n$} mit 
                     \textcolor{NavyBlue}{$\xi_i\in[x_{i-1},x_i]$},\,
                     \textcolor{NavyBlue}{$P=\{x_0,...,x_n\}$} 
                     \textcolor{NavyBlue}{
                     $|A-\sum_{i+1}^nf(\xi_i)(x_i-x_{i-1})|
                     <\varepsilon$}.
       \subsection{Integrierbare Funktionen}
              \colorbox{magenta}{Satz 5.2.1} 
                     Seien \textcolor{NavyBlue}{$f,g:[a,b]\longrightarrow\R$} 
                     \textbf{beschränkt, integrierbar und} 
                     \textcolor{NavyBlue}{$\lambda\in\R$}. 
                     Dann sind \textcolor{NavyBlue}{$f+g$},\,
                     \textcolor{NavyBlue}{$\lambda\cdot f$},\,
                     \textcolor{NavyBlue}{$f\cdot g$},\,
                     \textcolor{NavyBlue}{$|f|$},\,
                     \textcolor{NavyBlue}{$\max(f,g)$},
                     \textcolor{NavyBlue}{$\min(f,g)$} und (falls 
                     \textcolor{NavyBlue}{
                     $|g(x)\geqslant\beta>0\quad\forall x\in[a,b]$})
                     \textcolor{NavyBlue}{$\frac{f}{g}$}
                     \textbf{integrierbar}.\\
              \colorbox{green}{Bemerkung 5.2.2} 
                     Sei \textcolor{NavyBlue}{$\phi:[c,d]\longrightarrow\R$} 
                     eine \textbf{beschränkte Funktion}. Dann ist 
                     $(*)$\textcolor{NavyBlue}{$\sup\limits_{x,y\in[c,c]}|\phi(x)-\phi(y)|
                     =\sup\limits_{x\in[c,d]}\phi(x)-\inf\limits_{x\in[c,d]}\phi(x)$}.
                     Einerseits gilt offensichtlich 
                     \textcolor{NavyBlue}{$\forall x,y\in[c,d]$}: 
                     \textcolor{NavyBlue}{$\phi(x)\leqslant\sup\limits_{[c,d]}\phi,$}
                     \textcolor{NavyBlue}{$\quad\phi\geqslant\inf\limits_{[c,d]}\phi$}
                     also ist \textcolor{NavyBlue}{
                     $\phi(x)-\phi(y)\leqslant\sup\limits_{[c,d]}\phi-\inf\limits_{[c,d]}\phi$}, 
                     woraus durch vertauschen von \textcolor{NavyBlue}{$x,y$} folgt: 
                     \textcolor{NavyBlue}{
                     $|\phi(x)-\phi(y)|\leqslant\sup\limits_{[c,d]}-\inf\limits_{[c,d]}\phi$}. 
                     Andererseits sei \textcolor{NavyBlue}{$\varepsilon>0$}. 
                     Dann gibt es \textcolor{NavyBlue}{$\xi\in[c,d]$}
                     und \textcolor{NavyBlue}{$\eta\in[c,d]$ $\phi(\xi)>\varepsilon$} und 
                     \textcolor{NavyBlue}{$\phi(\eta)<\inf\limits_{[c,d]}\phi+\varepsilon$} 
                     woraus 
                     \textcolor{NavyBlue}{$\phi(\xi)-\phi(\eta)
                     >\sup\limits_{[c,d]}\phi-\inf\limits_{[c,d]}\phi-2\varepsilon$}
                     folgt. Dies zeigt die Aussage $(*)$ \\
              \colorbox{BurntOrange}{Korollar 5.2.3} 
                     Seien \textcolor{NavyBlue}{$P,Q$} Polynome und 
                     \textcolor{NavyBlue}{$[a,b]$}
                     ein \textbf{Intervall} in dem \textcolor{NavyBlue}{$Q$} 
                     \textbf{keine Nullstelle} besitzt. Dann ist 
                     \textcolor{NavyBlue}{
                     $[a,b]\longrightarrow\R,\,x\mapsto\frac{P(x)}{Q(x)}$}
                     \textbf{integrierbar}.\\
              \colorbox{cyan}{Definition 5.2.4} 
                     Eine Funktion 
                     \textcolor{NavyBlue}{$f:D\longrightarrow\R,\,D\subseteq\R$}
                     ist in \textcolor{NavyBlue}{$D$} 
                     \textbf{gleichmässig stetig}, wenn 
                     \textcolor{NavyBlue}{
                     $\forall\varepsilon>0\,\enspace\exists\delta>0$\quad
                     $\forall x,y\in D:\qquad|x-y|<\delta
                     \Longrightarrow|f(x)-f(y)|<\varepsilon$}.\\
              \colorbox{magenta}{Satz 5.2.6} 
                     (Heine 1872) Sei 
                     \textcolor{NavyBlue}{$f:[a,b]\longrightarrow\R$}
                     \textbf{stetig} in dem kompakten Intervall 
                     \textcolor{NavyBlue}{$[a,b]$}. 
                     Dann ist \textcolor{NavyBlue}{$f$} in 
                     \textcolor{NavyBlue}{$[a,b]$}
                     \textbf{gleichmässig stetig}.\\
              \colorbox{magenta}{Satz 5.2.7} 
                     Sei \textcolor{NavyBlue}{$f:[a,b]\longrightarrow\R$} 
                     \textbf{stetig}. So ist \textcolor{NavyBlue}{$f$} 
                     \textbf{integrierbar}.\\
              \colorbox{magenta}{Satz 5.2.8} 
                     Sei \textcolor{NavyBlue}{$f:[a,b]\longrightarrow\R$} 
                     \textbf{monoton}. So ist \textcolor{NavyBlue}{$f$} 
                     \textbf{integrierbar}.\\
              \colorbox{green}{Bemerkung 5.2.9} 
                     Seien \textcolor{NavyBlue}{$a<b<c$} und 
                     \textcolor{NavyBlue}{$f:[a,c]\longrightarrow\R$} 
                     \textbf{beschränkt} mit 
                     \textcolor{NavyBlue}{$f|_{[a,b]}$} und 
                     \textcolor{NavyBlue}{$f|_{[b,c]}$} 
                     \textbf{integrierbar}. Dann ist 
                     \textcolor{NavyBlue}{$f$} integrierbar und $(*)$ 
                     \textcolor{NavyBlue}{
                     $\int_a^cf(x)\,dx=\int_a^bf(x)\,dx+\int_b^cf(x)\,dx$}.
                     In der Tat ergibt die \textbf{Summe einer Obersumme 
                     (respektive Untersumme)} für 
                     \textcolor{NavyBlue}{$f|_{[a,b]}$} und 
                     \textcolor{NavyBlue}{$f|_{[b,c]}$} 
                     \textbf{eine Obersumme (respektive Untersumme)} für 
                     \textcolor{NavyBlue}{$f$}. Wir
                     erweitern jetzt die Definition von 
                     \textcolor{NavyBlue}{$\int_a^bf(x)\,dx$} auf: 
                     \textcolor{NavyBlue}{$\int_a^af(x)\,dx=0$} und wenn 
                     \textcolor{NavyBlue}{$a<b,\,\int_b^af(x)\,dx:=-\int_a^bf(x)\,dx$}.
                     Dann gilt $(*)$ für alle Tripel \textcolor{NavyBlue}{$a,b,c$} 
                     unter den entsprechenden Integrabilitätsvoraussetzungen. \\
              \colorbox{magenta}{Satz 5.2.10}
                     Sei \textcolor{NavyBlue}{$I\subsetneq\R$} 
                     ein \textbf{kompaktes Intervall} 
                     mit \textbf{Endpunkten} \textcolor{NavyBlue}{$a,b$} sowie 
                     \textcolor{NavyBlue}{$f_1,f_2:I\longrightarrow\R$}
                     \textbf{beschränkt 
                     integrierbar} und \textcolor{NavyBlue}{$\lambda_1,\lambda_2\in\R$}. 
                     Dann gilt: 
                     \textcolor{NavyBlue}{
                     $\int_a^b(\lambda_1f_1(x)
                     +\lambda_2f_2(x))\,dx
                     =\lambda_1\int_1^bf_1(x)\,dx
                     +\lambda_2\int_a^bf_2(x)\,dx$}.
       \subsection{Ungleichungen \& Mittelwertsatz}
              \colorbox{magenta}{Satz 5.3.1} 
                     Seien \textcolor{NavyBlue}{$f,g:[a,b]\longrightarrow\R$}
                     \textbf{beschränkt integrierbar}, und 
                     \textcolor{NavyBlue}{
                     $f(x)\leqslant g(x)\quad\forall x\in[a,b]$}.
                     Dann folgt: \textcolor{NavyBlue}{
                     $\int_a^bf(x)\,dx\leqslant\int_a^bg(x)\,dx$}. \\
              \colorbox{BurntOrange}{Korollar 5.3.2} 
                     Wenn \textcolor{NavyBlue}{$f:[a,b]\longrightarrow\R$} 
                     \textbf{beschränkt integrierbar}, folgt: 
                     \textcolor{NavyBlue}{
                     $|\int_a^bf(x)\,dx|\leqslant\int_a^b|f(x)|\,dx$}.\\
              \colorbox{magenta}{Satz 5.3.3} 
              (\textbf{Cauchy-Schwarz Ungleichung}) 
                     Seien \textcolor{NavyBlue}{$f,g:[a,b]\longrightarrow\R$}
                     \textbf{beschränkt integrierbar}. Dann gilt: 
                     \textcolor{NavyBlue}{$|\int_a^bf(x)g(x)\,dx|
                     \leqslant\sqrt{\int_a^bf^2(x)\,dx}
                     \sqrt{\int_a^bg^2(x)\,dx}$}.\\
              \colorbox{magenta}{Satz 5.3.4} 
              (\textbf{Mittelwertsatz}, Cauchy 1821) 
                     Sei \textcolor{NavyBlue}{$f:[a,b]\longrightarrow\R$} 
                     \textbf{stetig}. So 
                     \textcolor{NavyBlue}{$\exists\xi\in[a,b]$}:\, 
                     \textcolor{NavyBlue}{$\int_a^bf(x)\,dx=f(\xi)(b-a)$}.\\
              \colorbox{magenta}{Satz 5.3.6} (Cauchy 1821) Seien 
                     \textcolor{NavyBlue}{$f,g:[a,b]\longrightarrow\R$} wobei 
                     \textcolor{NavyBlue}{$f$} \textbf{stetig}, 
                     \textcolor{NavyBlue}{$g$} \textbf{beschränkt} und 
                     \textbf{integrierbar} mit 
                     \textcolor{NavyBlue}{$g\geqslant0\quad\forall x\in[a,b]$}. 
                     Dann gibt es \textcolor{NavyBlue}{$\xi\in[a,b]$} mit: 
                     \textcolor{NavyBlue}{
                     $\int_a^bf(x)g(x)\,dx
                     =f(\xi)\int_a^bg(x)\,dx$}.
       \subsection{Fundamentalsatz Differentialrechnung}
              \colorbox{magenta}{Satz 5.4.1}
              (\textbf{Fundamentalsatz der Analysis}) 
                     Seien \textcolor{NavyBlue}{$a<b$} und 
                     \textcolor{NavyBlue}{$f:[a,b]\longrightarrow\R$} 
                     \textbf{stetig}. Die Funktion 
                     \textcolor{NavyBlue}{
                     $F(x)=\int_a^xf(t)\,dt,\quad a\leqslant x\leqslant b$} 
                     ist in \textcolor{NavyBlue}{$[a,b]$} 
                     \textbf{stetig differenzierbar} 
                     und \textcolor{NavyBlue}{
                     $F'(x)=f(x)\quad\forall x\in[a,b]$}.\\
              \colorbox{cyan}{Definition 5.4.2} 
                     Sei \textcolor{NavyBlue}{$a<b$} und 
                     \textcolor{NavyBlue}{$f:[a,b]\longrightarrow\R$} \textbf{stetig}. 
                     Eine Funktion 
                     \textcolor{NavyBlue}{$F:[a,b]\longrightarrow\R$} heisst 
                     \textbf{Stammfunktion} von 
                     \textcolor{NavyBlue}{$f$}, wenn \textcolor{NavyBlue}{$F$} 
                     \textbf{(stetig) differenzierbar} in 
                     \textcolor{NavyBlue}{$[a,b]$}
                     ist und \textcolor{NavyBlue}{$F'=f$} in 
                     \textcolor{NavyBlue}{$[a,b]$} gilt.\\
              \colorbox{magenta}{Satz 5.4.3} 
              (\textbf{Fundamentalsatz der Differentialrechnung}) 
                     Sei \textcolor{NavyBlue}{$f:[a,b]\longrightarrow\R$} 
                     \textbf{stetig}. So gibt es eine 
                     \textbf{Stammfunktion} \textcolor{NavyBlue}{$F$} von \textcolor{NavyBlue}{$f$},
                     die bis auf eine addidive Konstante \textbf{eindeutig} 
                     bestimmt ist und: 
                     \textcolor{NavyBlue}{$\int_a^bf(x)\,dx=F(b)-F(a)$}. \\
              \colorbox{magenta}{Satz 5.4.5} 
              (\textbf{Partielle Integration}) 
                     Seien \textcolor{NavyBlue}{$a<b\in\R$} und 
                     \textcolor{NavyBlue}{$f,g:[a,b]\longrightarrow\R$}
                     \textbf{stetig differenzierbar}. Dann gilt: 
                     \textcolor{NavyBlue}{$\int_a^bf(x)g'(x)\,dx
                     =f(b)g(b)-f(a)g(a)-\int_a^bf'(x)g(x)\,dx$}. \\
              \colorbox{magenta}{Satz 5.4.6} 
              (\textbf{Substitution}) 
                     Sei \textcolor{NavyBlue}{$a<b$},\,
                     \textcolor{NavyBlue}{$\phi:[a,b]\longrightarrow\R$}
                     \textbf{stetig differenzierbar}, 
                     \textcolor{NavyBlue}{$I\subseteq\R$}
                     ein Intervall mit 
                     \textcolor{NavyBlue}{$\phi([a,b])\subseteq I$}
                     und \textcolor{NavyBlue}{$f:I\longrightarrow\R$} eine 
                     \textbf{stetige Funktion}. Dann gilt: 
                     \textcolor{NavyBlue}{$\int_{\phi(a)}^{\phi(b)}f(x)\,dx
                     =\int_a^bf(\phi(t))\phi'(t)dt$}.\\
              \colorbox{BurntOrange}{Korollar 5.4.8} 
                     Sei \textcolor{NavyBlue}{$I\subseteq\R$}
                     ein Intervall und \textcolor{NavyBlue}{$f:I\longrightarrow\R$} 
                     \textbf{stetig}. \\
                     \colorbox{SkyBlue}{(1)} Seien \textcolor{NavyBlue}{$a,b,c\in\R$}, 
                     so dass das abgeschlossene Intervall 
                     mit \textbf{Endpunkten} \textcolor{NavyBlue}{$a+c,b+c\in I$}. 
                     Dann gilt: 
                     \textcolor{NavyBlue}{
                     $\int_{a+c}^{b+c}f(x)\,dx=\int_a^bf(t+c)\,dt$}.\\
                     \colorbox{SkyBlue}{(2)} Seien \textcolor{NavyBlue}{$a,b,c\in\R$} mit 
                     \textcolor{NavyBlue}{$c\neq0$}, so dass das abgeschlossene Intervall mit den 
                     \textbf{Endpunkten} \textcolor{NavyBlue}{$ac,bc\in I$}. 
                     Dann gilt: \textcolor{NavyBlue}{
                     $\int_a^bf(ct)\,dt
                     =\frac{1}{c}\int_{ac}^{bc}f(x)\,dx$}.
       \subsection{Integration konvergenter Reihen}
              \colorbox{magenta}{Satz 5.5.1} 
                     Sei \textcolor{NavyBlue}{$f_n:[a,b]\longrightarrow\R$} 
                     eine Folge von \textbf{beschränkten, integrierbaren Funktionen}, 
                     die \textbf{gleichmässig} gegen 
                     eine Funktion 
                     \textcolor{NavyBlue}{$f:[a,b]\longrightarrow\R$} 
                     \textbf{konvergiert}.
                     So ist \textcolor{NavyBlue}{$f$} 
                     \textbf{beschränkt integrierbar und} 
                     \textcolor{NavyBlue}{
                     $\lim\limits_{n\to\infty}\int_a^bf_n(x)\,dx =\int_a^bf(x)\,dx$}.\\
              \colorbox{BurntOrange}{Korollar 5.5.2} Sei 
                     \textcolor{NavyBlue}{$f_n:[a,b]\longrightarrow\R$} eine Folge 
                     \textbf{beschränkter, integrierbarer Funktionen}, so dass 
                     \textcolor{NavyBlue}{$\sum_{n=0}^\infty f_n$} auf 
                     \textcolor{NavyBlue}{$[a,b]$}
                     \textbf{gleichmässig konvergiert}. Dann gilt: 
                     \textcolor{NavyBlue}{$\sum_{n=0}^\infty\int_a^bf_n(x)\,dx
                     =\int_a^b(\sum_{n=0}^\infty f_n(x))\,dx$}. \\
              \colorbox{BurntOrange}{Korollar 5.5.3} 
                     Sei \textcolor{NavyBlue}{$f(x)=\sum_{n=0}^\infty c_kx^k$} 
                     eine \textbf{Potenzreihe} mit positivem Konvergenzradius 
                     \textcolor{NavyBlue}{$\rho>0$}. Dann ist für jedes 
                     \textcolor{NavyBlue}{$0\leqslant r<\rho$},\,
                     \textcolor{NavyBlue}{$f$} 
                     auf \textcolor{NavyBlue}{$[-r,r]$}
                     \textbf{integrierbar} und es gilt 
                     \textcolor{NavyBlue}{$\forall x\in]-\rho,\rho[:\quad
                     \int_0^xf(t)\,dt
                     =\sum_{n=0}^\infty\frac{c_n}{n+1}x^{n+1}$}.
       % \colorbox{cyan}{Definition}  5.6.1 $\forall k\geqslant0$ ist das $k$'te Bernoulli 
       % Polynom $B_k(x)=k!P_k(x)$. 
       % Bernoulli Plynome: $B_0(x)=1\quad B_1(x)=x-\frac{1}{2}\quad B_2(x)=x^2-x+\frac{1}{6}$.
       % \colorbox{cyan}{Definition}  5.6.2 Sei $B_0=1$. Für alle $k\geqslant2$ definieren wir 
       % $B_{k-1}$ rekursiv: $\sum_{i=0}^{k-1}\binom{k}{i}B_i=0$.
       % \colorbox{magenta}{Satz} 5.6.3 $B_K(x)=\sum_{i=0}^k\binom{k}{i}B_ix^{k-i}$
       % \colorbox{green}{Bemerkung}     5.6.4 Für $k\geqslant2$: 
       % $B_k\colorbox{SkyBlue}{(1)}=\sum_{i=0}^k\binom{k}{i}B_i=\sum_{i=0}^{k-1}B_i+B_k=B_k=B_k(0)$
       %  Zur Aussage der Summationsformel definieren wir für $k\geqslant1$: 
       % $\widetilde{B}_k:[0,\infty[\longrightarrow\R$ als 
       %  $\widetilde{B}_k(x)=B_k(x)\enspace$ für $0\leqslant x<1$,
       % \qquad$\widetilde{B}_k(x)=B_k(x-n)\enspace$ für $n\leqslant x<n+1$ wobei $n\geqslant1$
       % \colorbox{magenta}{Satz} 5.6.5 Sei $f:[0,n]\longrightarrow\R\,k$-mal 
       % stetig differenzierbar, $k\geqslant1$. Dann gilt:
       %  \colorbox{SkyBlue}{(1)} Für $k=1$: 
       % $\sum_{i=1}^nf\colorbox{SkyBlue}{(i)}=\int_0^nf(x)\,dx+\frac{1}{2}(f(n)-f(0))
       % +\int_0^n\widetilde{B}_1(x)f'(x)\,dx$ 
       %  \colorbox{SkyBlue}{(2)} 
       % Stirling'sche Formel 
       % \colorbox{magenta}{Satz} 5.7.1 
       % \colorbox{yellow}{Lemma}   5.7.2 
       \stepcounter{subsection}
       \stepcounter{subsection}
       \subsection{Uneigentliche Integrale}
              \colorbox{cyan}{Definition 5.8.1} 
                     Sei \textcolor{NavyBlue}{$f:[a,\infty[\longrightarrow\R$}
                     \textbf{beschränkt und integrierbar} auf 
                     \textcolor{NavyBlue}{$[a,b]$} für alle 
                     \textcolor{NavyBlue}{$b>a$}. 
                     Wenn \textcolor{NavyBlue}{
                     $\lim\limits_{b\to\infty}\int_a^bf(x)\,dx$} 
                     existiert, bezeichnen wir den \textbf{Grenzwert} mit 
                     \textcolor{NavyBlue}{$\int_a^\infty f(x)\,dx$} 
                     und sagen, dass \textcolor{NavyBlue}{$f$} auf 
                     \textcolor{NavyBlue}{$[a,+\infty[$} 
                     \textbf{integrierbar ist}.\\
              \colorbox{yellow}{Lemma 5.8.3} 
                     Sei \textcolor{NavyBlue}{$f:[a,\infty[\longrightarrow\R$} 
                     \textbf{beschränkt und integrierbar auf} 
                     \textcolor{NavyBlue}{$[a,b]\enspace\forall b>a$}. 
                     \colorbox{SkyBlue}{(1)} Wenn 
                     \textcolor{NavyBlue}{$|f(x)|\leqslant g(x)\quad\forall x\geqslant a$}
                     und \textcolor{NavyBlue}{$g(x)$} auf \textcolor{NavyBlue}{$[a,\infty[$} 
                     \textbf{integrierbar} ist, ist 
                     \textcolor{NavyBlue}{$f$} auf \textcolor{NavyBlue}{$[a,\infty[$}
                     \textbf{integrierbar}. 
                     \colorbox{SkyBlue}{(2)} Wenn 
                     \textcolor{NavyBlue}{$0\leqslant g(x)\leqslant f(x)$} 
                     und \textcolor{NavyBlue}{$\int_a^\infty g(x)\,dx$} 
                     \textbf{divergiert}, \textbf{divergiert} auch 
                     \textcolor{NavyBlue}{$\int_a^\infty f(x)\,dx$}.\\
              \colorbox{magenta}{Satz 5.8.5} (McLaurin 1742) Sei 
                     \textcolor{NavyBlue}{$f:[1,\infty[\longrightarrow[0,\infty[$} 
                     \textbf{monoton fallend}. Die Reihe 
                     \textcolor{NavyBlue}{$\sum_{n=1}^\infty f(n)$} 
                     \textbf{konvergiert genau dann, wenn} 
                     \textcolor{NavyBlue}{$\int_1^\infty f(x)\,dx$} 
                     \textbf{konvergiert}. \\
              Eine Situation, die zu einem 
                     \colorbox{gray}{uneigentlichen Integral} führt, 
                     ist wenn \textcolor{NavyBlue}{$f:]a,b]\longrightarrow\R$} 
                     auf jedem Intervall 
                     \textcolor{NavyBlue}{$[a+\varepsilon,b],\enspace\varepsilon>0$} 
                     \textbf{beschränkt} und \textbf{integrierbar} ist, aber auf 
                     \textcolor{NavyBlue}{$]a,b]$} \textbf{nicht} notwendigerweise 
                     \textbf{beschränkt} ist. \\
              \colorbox{cyan}{Definition 5.8.8} In dieser Situation ist 
                     \textcolor{NavyBlue}{$f:]a,b]\longrightarrow\R$} 
                     \textbf{integrierbar}, wenn \textcolor{NavyBlue}{
                     $\lim\limits_{\varepsilon\to0^+}\int_{a+\varepsilon}^bf(x)\,dx$} 
                     existiert.In diesem Fall wird der \textbf{Grenzwert} mit 
                     \textcolor{NavyBlue}{$\int_a^bf(x)\,dx$} bezeichnet. \\
              \colorbox{cyan}{Definition 5.8.11} 
                     Für \textcolor{NavyBlue}{$s>0$} definieren wir 
                     \textcolor{NavyBlue}{
                     $\Gamma(s):=\int_0^\infty e^{-x}x^{s-1}\,dx$}. \\
              \colorbox{magenta}{Satz 5.8.12} 
              \textbf{(Bohr-Mollerup)} \\
                     \colorbox{SkyBlue}{(1)} 
                            Die \textbf{Gamma Funktion} erfüllt die Relationen:\\ 
                            \colorbox{SeaGreen}{(a)}
                                   \textcolor{NavyBlue}{$\Gamma(1)=1$} \qquad
                            \colorbox{SeaGreen}{(b)} 
                                   \textcolor{NavyBlue}{
                                   $\Gamma(s+1)=s\Gamma(s)\quad\forall s>0$} \\
                            \colorbox{SeaGreen}{(c)} 
                                   \textcolor{NavyBlue}{$\gamma$} ist 
                                   \textbf{logarithmisch konvex}, d.h. 
                                   \textcolor{NavyBlue}{
                                   $\Gamma(\lambda x+(1-\lambda)y)
                                   \leqslant\Gamma(x)^\lambda\Gamma(y)^{1--\lambda}$}
                                   für alle \textcolor{NavyBlue}{$x,y>0$} und 
                                   \textcolor{NavyBlue}{
                                   $0\leqslant\lambda\leqslant1$}.\\
                     \colorbox{SkyBlue}{(2)} Die \textbf{Gamma Funktion} ist die 
                            \textbf{einzige} Funktion 
                            \textcolor{NavyBlue}{$]0,\infty[\longrightarrow]0,\infty[$},
                            die (a), (b) und (c) erfüllt. 
                            Darüberhinaus gilt: 
                            \textcolor{NavyBlue}{$\Gamma(x)
                            =\lim\limits_{n\to+\infty}\frac{n!n^x}{x(x+1)...(x+n)}\quad
                            \forall x>0$}\\
              \colorbox{yellow}{Lemma 5.8.13} Sei \textcolor{NavyBlue}{$p>1$} und 
                     \textcolor{NavyBlue}{$q>1$} mit 
                     \textcolor{NavyBlue}{$\frac{1}{p}+{1}{q}=1$}. Dann gilt 
                     \textcolor{NavyBlue}{$\forall a,b\geqslant0$}: 
                     \textcolor{NavyBlue}{
                     $a\cdot b\leqslant\frac{a^p}{p}+\frac{b^q}{q}$}.\\
              \colorbox{magenta}{Satz 5.8.14} 
              (\textbf{Hölder Ungleichung}) 
                     Seien \textcolor{NavyBlue}{$p,q>1$} mit 
                     \textcolor{NavyBlue}{$\frac{1}{p}+\frac{1}{1}$}. Für 
                     \textbf{alle stetigen Funktionen} 
                     \textcolor{NavyBlue}{$f,g:[a,b]\longrightarrow\R$} gilt: 
                     \textcolor{NavyBlue}{
                     $\int_a^b|f(x)g(x)|\,dx\leqslant\norm{f}_p\norm{g}_q$}\\
       \subsection{Das unbestimmte Integral}
              \colorbox{magenta}{Satz 5.9.3} 
                     Seien \textcolor{NavyBlue}{$P,Q$} \textbf{Polynome} mit 
                     \textcolor{NavyBlue}{
                     $\operatorname{grad}(P)<\operatorname{grad}(Q)$} und 
                     \textcolor{NavyBlue}{$Q$}
                     mit \textbf{Produktzerlegung} $(*)$ Dann gibt es 
                     \textcolor{NavyBlue}{$A_{ij},B_{ij},C_{ij}\in\R$} mit: 
                     \textcolor{NavyBlue}{$\frac{P(x)}{Q(x)}
                     =\sum_{i=1}^l\sum_{j=1}^{m_i}
                     \frac{(A_{ij}+B_{ij}x)}{((x-a_i)^2+\beta_i^2)^j}
                     +\sum_{i=1}^k\sum_{j=1}^{n_i}\frac{C_{ij}}{x-\gamma_i)^j}$}.


\section{Anhang A}
              \colorbox{magenta}{Satz A.0.1} (\textbf{Binomialsatz}) 
                     \textcolor{NavyBlue}{$\forall x,y\in\C,\,n\geqslant1$} gilt: 
                     \textcolor{NavyBlue}{$(x+y)^n=\sum_{n}{k}x^ky^{n-k}$}.


\section{Wichtige Beispiele}
       \colorbox{gray}{Ungerade und gerade Funktionen} Sei \textcolor{NavyBlue}{$f(x)$}
              eine \textbf{gerade} Funktion. Dann: \textcolor{NavyBlue}{$f(x)=f(-x)$}.
              Sei \textcolor{NavyBlue}{$g(x)$} eine \textbf{ungerade} 
              Funktion. Dann: \textcolor{NavyBlue}{$-g(x)=g(-x)$}.
              Das \textbf{Produkt} von \textbf{2 geraden} Funktionen 
              ist \textbf{gerade}. Das \textbf{Produkt} von \textbf{2 
              ungeraden}
              Funktionen ist \textbf{gerade}. Das \textbf{Produkt einer 
              ungeraden und einer geraden} Funktion,
              ist \textbf{ungerade}.
              Für \textbf{ungerade} Funktionen gilt:
              \textcolor{NavyBlue}{$\int_{-a}^{+a}g(x)\,dx=0$}. (Dies kann man sich
              graphisch vorstellen).\\
       \colorbox{gray}{Konvergenztest für Reihen}
              Gegeben: \textcolor{NavyBlue}{$\sum_{n=0}^\infty a_n$}.\\
       \colorbox{SkyBlue}{(1)} Spezieller Typ?\\
              \colorbox{SeaGreen}{1.1} \textbf{Geometrische Reihe}: 
                     \textcolor{NavyBlue}{$\sum q^n$}? 
                     \textbf{Konvergent}, 
                     wenn: \textcolor{NavyBlue}{$|q|<1$}.\\
              \colorbox{SeaGreen}{1.2} \textbf{Alternierende Reihe}: 
                     \textcolor{NavyBlue}{$\sum(-1)^na_n$}? 
                     \textbf{Konvergent}, wenn:
                     \textcolor{NavyBlue}{$\lim a_n=0$}.\\
              \colorbox{SeaGreen}{1.3} \textbf{Riemann Zeta}: 
                     \textcolor{NavyBlue}{$\zeta(s)=\sum\frac{1}{n^s}$} 
                     \textbf{Konvergent}, wenn: \textcolor{NavyBlue}{$s>1$}.\\
              \colorbox{SeaGreen}{1.4} \textbf{Teleskopreihe} 
                     \textcolor{NavyBlue}{$\sum(b_n-b_{n-1})$}? 
                     \textbf{Konvergent}, wenn:\\
              \textcolor{NavyBlue}{$\lim b_n$} existiert.\\
       \colorbox{SkyBlue}{(2)} Kein spezieller Typ:\\
              \colorbox{SeaGreen}{2.1} \textcolor{NavyBlue}{$\lim a_n=0$}? 
                     Nein: \textbf{divergent}.\\
              \colorbox{SeaGreen}{2.2} \textbf{Quotientenkriterium} anwendbar?\\
              \colorbox{SeaGreen}{2.3} \textbf{Wurzelkriterium} anwendbar?\\
              \colorbox{SeaGreen}{2.4} Gibt es eine 
                     \textbf{konvergente Majorante}?\\
              \colorbox{SeaGreen}{2.5} Gibt es eine 
                     \textbf{divergente Minorante}?\\
              \colorbox{SeaGreen}{2.6} Nichts von all dem?\\
              $\Longrightarrow$ \textbf{kreativ sein}.\\
              % \colorbox{gray}{Stammfunktion von retionalen Funktionen} Sei
              %                 $R(x)=\frac{P(x)}{Q(x)}$ eine rationale Funktion.
              %                 $\int R(x)$ lästt sich als elementare Funktion darstellen,
              %                 d.h. als Funktion von Polynomen, rationalen, exponentiellen,
              %                 logarithmischen, trigonometrischen und inversen
              %                 trigonometrischen Funktionen.
              %                 Die Rechnung besteht dabei aus 3 Schritten:
              %                 1. Reduktion auf den Fall, wenn 
              %                 $\operatorname{grad}(P)<\operatorname{grad}(Q)$.
              %                 2. Zerlegung von $Q$ in lineare und quadratische Faktoren
              %                 sowie Partialbruchzerlegung von $R$.
              %                 3. Integration der Partialbrüche.
              %                 1. Wenn $\operatorname{grad}(P)\geqslant\operatorname{grad}(Q)$
              %                 wenden wir den euklidischen Algorithmus an:
              %                 $P(x)=S(x)Q(x)+\hat{P}(x)$ wobei 
              %                 $\operatorname{grad}(\hat{P})<\operatorname{grad}(Q)$.
              %                 Dann ist $\frac{P(x)}{Q(x)}=S(x)+\frac{\hat{P}(x)}{Q(x)}$
              %                 und eine Stammfunktion von $S(x)$ lässt sich leicht finden.
       \colorbox{gray}{Allgemeine Potenzen} Wir können die Exponentialfunktion
              und den natürlichen Logarithmus verwenden, um 
              allgemeine Potenzen zu definieren. Für $x>0$ und
              $a\in\R$ beliebig definieren wir:
              $x^a:=\exp(a\ln x)$. 
              Insbesondere: $x^0=1\enspace\forall x>0$.\\
       \colorbox{gray}{Trigonometrische Funktionen} Sinusfunktion für $z\in\C:$
              $\sin z=z-\frac{z^3}{3!}+\frac{z^5}{5!}-\frac{z^7}{7!}+\dots
              =\sum_{n=0}^\infty\frac{(-1)^nz^{2n+1}}{(2n+1)!}$.
              Kosinusfunktion für $z\in\C:$
              $\cos z=1-\frac{z^2}{2!}+\frac{z^4}{4!}-\frac{z^6}{6!}+\dots
              =\sum_{n=0}^\infty\frac{(-1)^nz^{2n}}{(2n+1)!}$.
              Tangensfunktion für $z\notin\frac{\pi}{2}+\pi\cdot\mathbb{Z}$:
              $\tan z=\frac{\sin z}{\cos z}$. \quad
              Cotangensfunktion für $z\notin\pi\cdot\mathbb{Z}$:
              $\cot z=\frac{\cos z}{\sin z}$.\\
       \colorbox{gray}{Hyperbelfunktionen} $\forall x\in\R$:\quad
              $\cosh x=\frac{e^x+e^{-x}}{2}$.\quad
              $\sinh x=\frac{e^x-e^{-x}}{2}$.\quad
              $\tanh x=\frac{\sinh x}{\cosh x}=\frac{e^x-e^{-x}}{e^x+e^{-x}}$.
              Es gilt offensichtlich:
              $\cosh x\geqslant1\enspace\forall x\in\R$,
              $\sinh x\geqslant1\enspace\forall x\in]0,+\infty[$,
              $\sin(0)=0$. Daraus folgt: $\cosh$ ist 
              auf $[0,\infty[$ strikt
              monoton wachsend, $\cosh(0)=1$ und
              $\lim\limits_{x\to+\infty}\cosh x=+\infty$. Also ist
              $\cosh:[0,\infty[\longrightarrow[1,\infty[$ 
              bijektiv. Deren Umkehrfunktion wird mit 
              $\operatorname{arcosh}:[1,\infty[\longrightarrow[0,\infty[$ 
              bezeichnet. Unter Verwendung von
              $\cosh^2x-\sinh^2x=1\enspace\forall x\in\R$ 
              folgt:
              $\operatorname{arcosh'y}
              =\frac{1}{\sqrt{y^2-1}}\enspace\forall y\in]1,+\infty[$.
              Analog zeigt man, dass 
              $\sinh:\R\longrightarrow\R$ streng monoton
              wachsend und bijektiv ist. Dessen Umkehrfunktion 
              wird mit 
              $\operatorname{arsinh}:\R\longrightarrow\R$ 
              bezichnet und es gilt:
              $\operatorname{arsinh}'y
              =\frac{1}{\sqrt{1+y^2}}\enspace\forall y\in\R$. 
              Für $\tanh x$ folgt:
              $\tanh'x=\frac{1}{\cosh^2x}>0$ Also ist tanh auf R streng
              monoton wachsend und man zeigt, dass 
              $\lim\limits_{x\to+\infty}\tanh x=1$,
              $\lim\limits_{x\to-\infty}\tanh x=-1$.
              Die Funktion $\tanh:\R\longrightarrow]-1,1[$ ist
              bijektiv. Ihre Umkehrfunktion 
              wird mit 
              $\operatorname{artanh}:]-1,1[\longrightarrow\R$
              bezeichnet. Es gilt dann:
              $\operatorname{artanh}'y=\frac{1}{1-y^2}\enspace
              \forall y\in]-1,1[$. 


\subsection{Ableitungen}
$(ax^z)'=azx^{z-1}$\\
$(x^x)'=(e^{x\ln x})'=(\ln(x)+1)e^x$\\
$(x\ln x)'=\ln(x)+1$\\
$e'^x=e^x$\\
$\sin'x=\cos x$\\
$\cos'x=-\sin x$\\
$\tan'x=\frac{1}{\cos^2x}$\\
$\cot'x=-\frac{1}{\sin^2x}$\\
$\ln'x=\frac{1}{x}$\\
$\arcsin'x=\frac{1}{\sqrt{1-x^2}}$\\
$\arccos'x=\frac{-1}{\sqrt{1-x^2}}$\\
$\arctan'x=\frac{1}{1+x^2}$\\
$\sinh'x=\cosh x$\\
$\cosh'x=\sinh x$\\
$\tanh'x=\frac{1}{\cosh^2x}$\\
$\operatorname{arsinh}'y
        =\frac{1}{\sqrt{1+y^2}}\quad\forall y\in\R$\\
$\operatorname{arcosh'y}
        =\frac{1}{\sqrt{y^2-1}}\quad\forall y\in]1,+\infty[$\\
$\operatorname{artanh}'y=\frac{1}{1-y^2}\quad\forall y\in]-1,1[$


\subsection{Integrale}
$\int x^s\,dx=\frac{x^{s+1}}{s+1}+C\quad s\neq-1$\qquad
                $\int x^s\,dx=\ln x+C\quad s=-1$\\
$\int\sin x\,dx=-\cos x+C$\\
$\int\cos x\,dx=\sin x+C$\\
$\int\sinh x\,dx=\cosh x+C$\\
$\int\cosh x\,dx=\sinh x+C$\\
$\int\frac{1}{\sqrt{1-x^2}}\,dx=\arcsin x+C$\\
$\int\frac{1}{\sqrt{1+x^2}}\,dx=\operatorname{arsinh} x+C$\\
$\int\frac{1}{1+x^2}\,dx=\arctan x+C$\\
$\int\frac{1}{\sqrt{x^2-1}}\,dx=\operatorname{arcosh}+C$\\
$\int e^x\,dx=e^x+C$\\
$\int\ln x\,dx=x\ln x-x+C$ (verwende $\ln x=\ln x\cdot1$)\\
$\int x\ln x\,dx=\frac{x^2}{2}\ln x-\frac{x^2}{4}+C$\\
$\int x^2\sin x\,dx=-x^2\cos x+2x\sin x+2\cos x$\\
$n\geqslant1\quad I_n=\int\sin^nx\,dx=-\frac{1}{n}\cos x\sin^{n-1}x
                +\frac{n-1}{n}I_{n-2}$\\
$\int(ax+b)^s\,dx=\frac{1}{a(s+1)}(ax+b)^{s+1}+C\quad s\neq1$\qquad
                $\int(ax+b)^s\,dx=\frac{1}{a}\ln|ax+b|+C$

% ln x 
% arcsin x
% arccos x
% tan x 
% cot x 
% arctan x
% sinh x 
% cosh x 
% tanh x 
% $ax^z$ 

\subsection{Additionstheoreme}
$\sin(x\pm y)=\sin x\cos y\pm\cos x\sin y$\\
$\cos(x\pm y)=\cos x\cos y\mp\sin x\sin y$\\
$\tan(x\pm y)=\frac{\tan x\pm\tan y}{1\mp\tan x\tan y}$\\
$\sin 2x=2\sin x\cos x$\\
$\cos 2x=\cos^2x-\sin^2x=2\cos^2x-1=1-2\sin^2x$\\
$\tan 2x=\frac{2\tan x}{1-\tan^2x}$\\
$\sin 3x=3\sin x-4\sin^3x$\\
$\cos 3x=4\cos^3x-3\cos x$\\
$\tan 3x=\frac{3\tan x-\tan^3x}{1-3\tan^2 x}$\\
$\sin^2\frac{x}{2}=\frac{1-\cos x}{2}$\\
$\cos^2\frac{x}{2}=\frac{1+\cos x}{2}$\\
$\tan^2\frac{x}{2}=\frac{1-\cos x}{1+\cos x}$
$\tan\frac{x}{2}=\frac{1-\cos x}{\sin x}=\frac{\sin x}{1+\cos x}$\\
$\sin x+\sin y=2\sin\frac{x+y}{2}\cos\frac{x-y}{2}$\\
$\sin x-\sin y=2\cos\frac{x+y}{2}\sin\frac{x-y}{2}$\\
$\cos x+\cos y=2\cos\frac{x+y}{2}\cos\frac{x-y}{2}$\\
$\cos x-\cos y=-2\sin\frac{x+y}{2}\sin\frac{x-y}{2}$\\
$\sin x\sin y=\frac{1}{2}(\cos(x-y)-\cos(x+y))$\\
$\cos x\cos y=\frac{1}{2}(\cos(x-y)+\cos(x+y))$\\
$\sin x\cos y=\frac{1}{2}(\sin(x-y)+\sin(x+y))$

\subsection{Grenzwerte}
$\lim\limits_{x\to\infty}(1+\frac{x}{n})^n=e^x$\\
$\forall\alpha\in\R\quad\lim\limits_{x\to\infty}\sqrt[n]{n^\alpha}=1$\\
$\lim\limits_{x\to\infty}\sqrt[n]{n!}=\infty$\\
$\forall\alpha\in\R,\,|q|<1\quad\lim\limits_{x\to\infty}n^\alpha\cdot q^n=0$\\
$\lim\limits_{x\to0}\sqrt[x]x=...$\\
$\lim\limits_{x\to0}x^x=...$\\
\end{multicols}
\end{document}